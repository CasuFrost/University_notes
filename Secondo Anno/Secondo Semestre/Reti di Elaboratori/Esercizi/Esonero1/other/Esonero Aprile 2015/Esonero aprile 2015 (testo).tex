\documentclass[12pt, letterpaper]{article}
\usepackage{graphicx} % Required for inserting images
\usepackage{hyperref}
\usepackage{listings}
\usepackage{amssymb}
\usepackage{amsmath}
\usepackage[english]{babel}
\usepackage{nicefrac, xfrac}
\usepackage{mathtools}
\usepackage[table,xcdraw]{xcolor}
\definecolor{light-gray}{gray}{0.95}
\definecolor{sap}{RGB}{130, 36, 51}
\definecolor{lg}{RGB}{102, 161, 95}
\definecolor{g}{RGB}{60, 50, 50}
\usepackage[paper=a4paper,left=20mm,right=20mm,bottom=25mm,top=25mm]{geometry}
\newcommand{\code}[1]{\colorbox{light-gray}{\texttt{#1}}}
\newcommand{\shelll}[1]{\colorbox{black}{\textcolor{white}{\texttt{#1}}}}
\newcommand{\shell}[1]{\colorbox{black}{\textcolor{white}{\texttt{casufrost@debian:$\sim$\$ #1}}}}
\newcommand{\codee}[1]{\colorbox{white}{\texttt{#1}}}
\newcommand{\acc}{\\\hphantom{}\\}
\newcommand{\dete}{{\rightarrow}}
\newcommand{\fdot}{{\(\bullet\) }}
\newcommand{\comm}[1]{\color{lg}\textit{\hphantom{spaz}// \text{#1}}\color{black}}
\newcommand{\textg}[1]{\color{g}{\textbf{#1}}\color{black}}
\newcommand{\boxedMath}[1]{\begin{tabular}{|c|}\hline \texttt{#1} \\ \hline\end{tabular} :}
\title{Esonero Aprile 2015 - Reti di Elaboratori}
\date{\vspace{-5ex}}
\begin{document}
\maketitle
\textbf{1) Quali sono i livelli protcollari presenti su un host?}\acc 



\textbf{2) Quali sono i vantaggi di avere uno stack protocollare basato slla stratificazione (architettura a livelli)?}\acc 



\textbf{3) Come si calcola il ritardo di propagazione sperimentato nella trasmissione di un pacchetto di lunghezza $L$ bit, su un link 
di capacità $R$($\nicefrac{bit}{sec}$) e lungo $D$ chilometri?}\acc 


\textbf{4) Come si calcola il ritardo di trasmissione sperimentato nella trasmissione di un pacchetto
di lunghezza $L$ bit, su un link di capacità $R$($\nicefrac{bit}{sec}$) e lungo $D$ chilometri?}\acc 


\textbf{5) Quali sono le prestazioni in termini di latenza (ritardo) sperimentate su tale link nel caso di un 
processo di arrivo del traffico Poissoniano al crescere del carico offerto sul link?}\acc 


\textbf{6) Cosa è un socket? }\acc 


\textbf{7) Cosa identifica univocamente un socket TCP?}\acc 


\textbf{8) Quale livello di trasporto utilizza l'applicazione DNS e perché?}\acc 



\textbf{9) Quale protocollo è utilizzato per la comunicazione fra mail server?}\acc 



\textbf{10) Cosa si intende per controllo di flusso?}\acc 



\textbf{11) Cosa si intende per slow start in TCP?}\acc 



\textbf{12) La trasmissione affidabile dell'informazione in TPC avviene secondo un protocollo di AutomaticRepeat Request ibrido.
 Il protocollo è un mix di quali protocolli? (Quali caratteristiche ha di uno e dell'altro?)}\acc 



 \textbf{13) Quali livelli protocollari sono presenti su un router?}\\
 \color{red}Domanda riguardante argomenti
 non ancora presentati 
 nel programma oggi 07/04/2024\color{black}\acc
 Livello di rete e livello di collegamento (risposta "a sentimento").\acc 



 \textbf{14) Che tipo di protocollo di routing è BGP (intra o 
 inter-autonomoussystemrouting? Distancevector/Link State/Policy Based/PathBased?}\\
 \color{red}Domanda riguardante argomenti
 non ancora presentati 
 nel programma oggi 07/04/2024\color{black}\acc\hphantom{text}\acc
 \Large  \textbf{Domande estese}\acc
 \normalsize


 \textbf{1) Si descriva in dettaglio il protocollo RIP (in quale contesto è usato? Quale 
 è l'algoritmo usato per il calcolo dei cammini minimi? Quali la metrica? Quale il funzionamento 
 del protocollo in dettaglio? Quali i problemi del funzionamento del protocollo e come sono risolti in pratica)?}\\
 \color{red}Domanda riguardante argomenti
 non ancora presentati 
 nel programma oggi 07/04/2024\color{black}\acc\hphantom{text}\acc


 \textbf{2) Si descriva il funzionamento dell'applicazione Web(in quale livello e per cosa è usata?
  Descrizione del protocollo usato per scambiare i messaggi. Si spieghi cosa sono i cookies. Discutete 
  anche come si possono migliorare le prestazioni dell'applicazione web.}
\end{document}