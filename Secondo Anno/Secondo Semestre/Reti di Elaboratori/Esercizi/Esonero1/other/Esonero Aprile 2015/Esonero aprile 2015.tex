\documentclass[12pt, letterpaper]{article}
\usepackage{graphicx} % Required for inserting images
\usepackage{hyperref}
\usepackage{listings}
\usepackage{amssymb}
\usepackage{amsmath}
\usepackage[english]{babel}
\usepackage{nicefrac, xfrac}
\usepackage{mathtools}
\usepackage[table,xcdraw]{xcolor}
\definecolor{light-gray}{gray}{0.95}
\definecolor{sap}{RGB}{130, 36, 51}
\definecolor{lg}{RGB}{102, 161, 95}
\definecolor{g}{RGB}{60, 50, 50}
\usepackage[paper=a4paper,left=20mm,right=20mm,bottom=25mm,top=25mm]{geometry}
\newcommand{\code}[1]{\colorbox{light-gray}{\texttt{#1}}}
\newcommand{\shelll}[1]{\colorbox{black}{\textcolor{white}{\texttt{#1}}}}
\newcommand{\shell}[1]{\colorbox{black}{\textcolor{white}{\texttt{casufrost@debian:$\sim$\$ #1}}}}
\newcommand{\codee}[1]{\colorbox{white}{\texttt{#1}}}
\newcommand{\acc}{\\\hphantom{}\\}
\newcommand{\dete}{{\rightarrow}}
\newcommand{\fdot}{{\(\bullet\) }}
\newcommand{\comm}[1]{\color{lg}\textit{\hphantom{spaz}// \text{#1}}\color{black}}
\newcommand{\textg}[1]{\color{g}{\textbf{#1}}\color{black}}
\newcommand{\boxedMath}[1]{\begin{tabular}{|c|}\hline \texttt{#1} \\ \hline\end{tabular} :}
\title{Esonero Aprile 2015 - Reti di Elaboratori}
\date{\vspace{-5ex}}
\begin{document}
\maketitle
\textbf{1) Quali sono i livelli protcollari presenti su un host?}\acc 
I livelli protocollari presenti su un host sono : applicazione, TSL, trasporto e rete. Un host invia e riceve messaggi dal/al livello 
di applicazione, li incapsula in segmenti (trasporto) per scambiarli fra processi su diversi host, e li incapsula a livello 
di rete in datagrammi per far si che raggiungano l'host richiesto.\acc


\textbf{2) Quali sono i vantaggi di avere uno stack protocollare basato slla stratificazione (architettura a livelli)?}\acc 
I vantaggi derivano dal fatto che ogni singolo livello, può usufruire dei servizi implementati dal livello inferiore, i livelli 
vengono visti come delle blackbox, ciò garantisce una grande modularità e rende Internet più organizzato, anche se ciò aumenta la 
complessità della struttura ed introduce ridondanza.\acc 


\textbf{3) Come si calcola il ritardo di propagazione sperimentato nella trasmissione di un pacchetto di lunghezza $L$ bit, su un link 
di capacità $R$($\nicefrac{bit}{sec}$) e lungo $D$ chilometri?}\acc 
Il ritardo di propagazione è indipendente dal rate $R$ e dalla lunghezza $L$ del pacchetto, è il tempo che impiega un bit per 
propagarsi in un link, sia $v$ la velocità di propagazione di un bit sul link misurata in $\nicefrac{m}{s}$, il ritardo di propagazione è dato 
da $\frac{D\cdot 10^{3}}{v} sec$. (ho moltiplicato $D$ per $10^{3}$ per convertirlo da chilometri a metri).\acc 


\textbf{4) Come si calcola il ritardo di trasmissione sperimentato nella trasmissione di un pacchetto
di lunghezza $L$ bit, su un link di capacità $R$($\nicefrac{bit}{sec}$) e lungo $D$ chilometri?}\acc 
Il ritardo di trasmissione è esattamente $\frac{L}{R}sec$.\acc 


\textbf{5) Quali sono le prestazioni in termini di latenza (ritardo) sperimentate su tale link nel caso di un 
processo di arrivo del traffico Poissoniano al crescere del carico offerto sul link?}\acc 
Se consideriamo $\lambda$ come il tasso di arrivo dei pacchetti misurato in pacchetti al secondo, si ha che, il valore 
adimensionale $\nicefrac{\lambda\cdot L}{R}$ da una misura del tempo di attesa in coda di un pacchetto, se tale valore 
tende ad 1, il tempo di attesa tende ad $+\infty$, se tale valore è maggiore o uguale ad 1, il tempo di attesa è $+\infty$.\acc 


\textbf{6) Cosa è un socket? }\acc 
Un socket non è altro che il canale virtuale nella quale comunicano due processi su host diversi. Il livello di trasporto mette in 
comunicazione due processi tramite i numeri di porta, un socket è una mappatura numero di porta - processo. Più socket possono 
essere associati ad un unico processo.\acc 


\textbf{7) Cosa identifica univocamente un socket TCP?}\acc 
Un socket TCP è identificato univocamente dalla quadrupla : (IP mittente, porta mittente, IP destinatario, porta destinatario).\acc 


\textbf{8) Quale livello di trasporto utilizza l'applicazione DNS e perché?}\acc
I messaggi DNS devono essere i più veloci possibili, non necessitano di stabilire una connessione fra DNS server e colui che sta risolvendo 
un nome, è una richiesta unica di un record di un database, per questo, si utilizza UDP. Se il pacchetto con la richiesta viene perduto, 
la richiesta viene semplicemente ritrasmessa.\acc 

\textbf{9) Quale protocollo è utilizzato per la comunicazione fra mail server?}\acc 
Il protocollo SMTP.\acc 


\textbf{10) Cosa si intende per controllo di flusso?}\acc 
Quando due processi comunicano, può succedere che uno dei due processi, invii segmenti più rapidamente rispetto a quanto l'altro 
processo possa elaborarne ed inviarne al livello di applicazione, tale "ingolfamento" fa si che il buffer del socket del processo che 
sta ricevendo pacchetti venga riempito. Il controllo del flusso (implementato dal TCP), fa si che i processi forniscano informazioni 
sul numero di segmenti che possono ricevere senza venire ingolfati, facendo si che l'altro processo comunicante si adatti e non 
invii segmenti troppo rapidamente, evitando l'ingolfamento.\acc 


\textbf{11) Cosa si intende per slow start in TCP?}\acc 
Si intende il processo/algoritmo per il quale, colui che non ha informazioni sulla finestra di ricevimento del destinatario, 
inizia ad inviare segmenti, aumentando la propria finestra di invio in maniera esponenziale, in modo tale da raggiungere 
rapidamente il limite di segmenti inviabili, per poi ri-impostare correttamente tale finestra in base alle disponibilità del 
destinatario. Se tale algoritmo non venisse applicato, si impiegherebbe troppo tempo a raggiungere il limite prima citato, 
causando uno spreco di banda.\acc 


\textbf{12) La trasmissione affidabile dell'informazione in TPC avviene secondo un protocollo di AutomaticRepeat Request ibrido.
 Il protocollo è un mix di quali protocolli? (Quali caratteristiche ha di uno e dell'altro?)}\acc 
 Il TCP implementa un ibrido fra i due modelli di pipelining \textit{Go-back-$n$} e \textit{selective-repeat}, possono venire 
 inviati fino ad $n$ pacchetti alla volta, ed il ricevitore ha una finestra di $m$ pacchetti che può ricevere senza venire ingolfato, 
 gli ACK sono cumulativi, e vi è un solo timer alla volta che fa riferimento al segmento inviato meno recentemente della quale 
 non si è ancora ricevuto un ACK. La ritrasmissione di un segmento avviene in due circostanze, la prima, è allo scadere del 
 timeout ad esso associato, la secondo, è alla ricezione di 3 ACK duplicati da parte del destinatario (fast retransmit).\acc 


 \textbf{13) Quali livelli protocollari sono presenti su un router?}\\
 \color{red}Domanda riguardante argomenti
 non ancora presentati 
 nel programma oggi 07/04/2024\color{black}\acc
 Livello di rete e livello di collegamento (risposta "a sentimento").\acc 



 \textbf{14) Che tipo di protocollo di routing è BGP (intra o 
 inter-autonomoussystemrouting? Distancevector/Link State/Policy Based/PathBased?}\\
 \color{red}Domanda riguardante argomenti
 non ancora presentati 
 nel programma oggi 07/04/2024\color{black}\acc\hphantom{text}\acc
 \Large  \textbf{Domande estese}\acc
 \normalsize


 \textbf{1) Si descriva in dettaglio il protocollo RIP (in quale contesto è usato? Quale 
 è l'algoritmo usato per il calcolo dei cammini minimi? Quali la metrica? Quale il funzionamento 
 del protocollo in dettaglio? Quali i problemi del funzionamento del protocollo e come sono risolti in pratica)?}\\
 \color{red}Domanda riguardante argomenti
 non ancora presentati 
 nel programma oggi 07/04/2024\color{black}\acc\hphantom{text}\acc


 \textbf{2) Si descriva il funzionamento dell'applicazione Web(in quale livello e per cosa è usata?
  Descrizione del protocollo usato per scambiare i messaggi. Si spieghi cosa sono i cookies. Discutete 
  anche come si possono migliorare le prestazioni dell'applicazione web.}\acc 
Un applicazione web. è un'applicazione/processo che usufruisce delle funzionalità di Internet, ed è definita 
nel livello di applicazione. Un host può fornire un applicazione web tramite il protocollo HTTP, creando un 
server HTTP, aprendo un socket (TCP) sulla porta 80, necessario per accogliere le richieste. Un client HTTP, 
farà una richiesta al server, chiedendo degli oggetti web, che sono dei file presenti sul server. Per ogni file richiesto 
da un client HTTP, si apre una connessione TCP, sulla quale verrà trasferito il file, per poi venire chiusa alla fine della 
condivisione. Un oggetto web, può fare riferimento ad altri oggetti web, se si richiede il primo, dovranno essere 
trasferiti anche i restanti oggetti referenziati. Una comunicazione fra client e server HTTP non è persistente, dato 
che alla fine di uno scambio di un oggetto la connessione viene chiusa, per questo, il client può salvare dei file 
locali gestiti dal browser detti cookies, che permettono al server di avere informazioni riguardanti colui 
che sta facendo le richieste, in quanto queste conterranno appunto tali cookies. 
È possibile migliorare le prestazioni dell'applicazione, facendo si che gli oggetti HTTP vengano divisi in frame 
più piccoli ed interlacciati, in modo tale che l'ordine di invio degli oggetti non sia identico all'ordine di 
richiesta (se viene richiesto un oggetto molto grande per primo, tutti i restanti oggetti non saranno disponibili
fino alla fine della condivisione di questo).  
\end{document}