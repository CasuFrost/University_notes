\documentclass{article}
\usepackage{graphicx} % Required for inserting images
\usepackage{amsmath}
\usepackage[margin=1in]{geometry}
\usepackage{epigraph}
\usepackage{hyperref}

\title{Azienda 1 (gruppo 42)}
% \author{ Giacomo Biribicchi \and Marco Casu \and Ionut Cicio \and Christian Di Manno \and Alessandro Gautieri }
\date{}


\begin{document}

\maketitle

 \epigraph{The answer to the Ultimate Question of Life, the Universe, and Everything}{\textit{Douglas Adams \\ "The Hitchhiker's Guide to the Galaxy"}}

% Si vuole sviluppare un sistema informativo per la gestione dei dati sul personale di una certa azienda costituita da diversi dipartimenti. Durante la fase di raccolta dei requisiti è stata prodotta la specifica dei requisiti mostrata di seguito. Si chiede di iniziare la fase di Analisi dei requisiti ed in particolare di:

% raffinare la specifica dei requisiti eliminando inconsistenze, omissioni o ridondanze e produrre un elenco numerato di requisiti il meno ambiguo possibile
% produrre un diagramma UML delle classi concettuale che modelli i dati di interesse, utilizzando solo i costrutti di classe, associazione, attributo

\section{Requisiti}

I dati di interesse per il sistema sono \textbf{impiegati}, \textbf{dipartimenti}, \textbf{direttori} dei dipartimenti e \textbf{progetti} aziendali.

\paragraph{Impiegato}

% \textit{(se vogliamo essere pedanti, l'impiegato deve avere almeno 16 anni e al più 70, ma dipende dal ma dipende dal giorno in cui viene assunto, quindi è un requisito inutile)}

\begin{enumerate}
    \item Nome (include anche il doppio nome), cognome, data di nascita e stipendio attuale. 
    % \textit{(dato che interessa solo lo stipendio attuale, non serve memorizzare lo storico degli stipendi)}
    \item Dipartimento a cui afferisce, con relativa data di afferenza \textit{(esattamente uno)}.
    \item La data di afferenza al dipartimento non può essere precedente alla creazione dell'azienda.
    \item Ogni dipendente può cambiare dipartimento, e vogliamo memorizzare lo storico dei dipartimenti in cui ha lavorato.
    \item L'afferenza con la data più recente sarà considerata 
    quella corrente.
    \item Un impiegato può dirigere più di un dipartimento.
    \item Un dirigente non afferisce al dipartimento che dirige. Afferisce al dipartimento riservato ai dirigenti.
    \item L'impiegato può partecipare a \textbf{0 o più progetti} \textit{(un nuovo impiegato non partecipa ancora a nessun progetto)}.
\end{enumerate}

\paragraph{Dipartimento}

\begin{enumerate}
    \item Nome, telefono.
\end{enumerate}

\paragraph{Progetto}

\begin{enumerate}
    \item Ad un progetto possono partecipare \textbf{0 o più impiegati} \textit{(potrebbe succedere che il progetto è nuovo e non è stato assegnato nessun impiegato)}.
    \item Il budget di un progetto non può essere negativo. 
\end{enumerate}

% I dati di interesse per il sistema sono impiegati, dipartimenti, direttori dei dipartimenti e
% progetti aziendali.
% Di ogni impiegato interessa conoscere il nome, il cognome, la data di nascita e lo
% stipendio attuale, il dipartimento (esattamente uno) al quale afferisce.
% Di ogni dipartimento interessa conoscere il nome, il numero di telefono del centralino,
% e la data di afferenza di ognuno degli impiegati che vi lavorano.
% Di ogni dipartimento interessa conoscere inoltre il direttore, che è uno degli impiegati
% dell’azienda.
% Il sistema deve permettere di rappresentare i progetti aziendali nei quali sono coinvolti
% i diversi impiegati. Di ogni progetto interessa il nome ed il budget. Ogni impiegato può
% partecipare ad un numero qualsiasi di progetti.

% Di ogni impiegato interessa conoscere il nome, il cognome, la data di nascita, lo stipendio attuale e il dipartimento al quale afferisce (\textit{esattamente uno}, con la rispettiva data di afferenza).
% Di ogni dipartimento interessa conoscere il nome, il numero di telefono del centralino
% Di ogni dipartimento interessa conoscere inoltre il direttore, che è uno degli impiegati dell’azienda.
% Il sistema deve permettere di rappresentare i progetti aziendali nei quali sono coinvolti i diversi impiegati.
% Di ogni progetto interessa il nome ed il budget.
% Ogni impiegato può partecipare ad un numero qualsiasi di progetti.

\section{UML}

\includegraphics[width=\textwidth]{UML.png}


\end{document}

