\documentclass[12pt, letterpaper]{article}
\usepackage{graphicx} % Required for inserting images
\usepackage{hyperref}
\usepackage{listings}
\usepackage{amssymb}
\usepackage{amsmath}
\usepackage[english]{babel}
\usepackage{nicefrac, xfrac}
\usepackage{mathtools}
\newcommand{\acc}{\\\hphantom{}\\}
\usepackage[table,xcdraw]{xcolor}
\usepackage[paper=a4paper,left=20mm,right=20mm,bottom=25mm,top=25mm]{geometry}
\renewcommand{\labelenumii}{\arabic{enumi}.\arabic{enumii}}
    \renewcommand{\labelenumiii}{\arabic{enumi}.\arabic{enumii}.\arabic{enumiii}}
    \renewcommand{\labelenumiv}{\arabic{enumi}.\arabic{enumii}.\arabic{enumiii}.\arabic{enumiv}}
\title{Voli aerei 1 (gruppo 42)}
% \author{ Giacomo Biribicchi \and Marco Casu \and Christian Di Manno \and Alessandro Gautieri }
\date{}


\begin{document}

\maketitle


\section{Requisiti}

I dati di interesse per il sistema sono \underline{Voli}, \underline{Compagnia} ed \underline{Aeroporti}.
\begin{enumerate}
    \item \textbf{Volo}\begin{enumerate}
        \item codice - Serie di caratteri alfanumerici che identificano univocamente un \underline{Volo}.
        \item durata - Si usa un dato composto, consistente in un intero positivo per identificare la durata in ore, e due interi 
        compresi fra 0 e 59, rispettivamente per i minuti ed i secondi.
        \item compagnia - La compagnia che fornisce il volo.
        \item aereoporto di partenza 
        \item aereoporto di arrivo
    \end{enumerate}
    \item \textbf{Aereoporto}\begin{enumerate}
        \item codice - Serie di caratteri alfanumerici che identificano univocamente un \underline{Aereoporto}.
        \item nome 
        \item città - La città in cui risiede l'\underline{Aereoporto}.
        \item nazione - La nazione in cui si trova la città.
    \end{enumerate}
    \item \textbf{Compagnia}\begin{enumerate}
        \item nome 
        \item anno di fondazione 
        \item città sede direzione
    \end{enumerate}
\end{enumerate}
Saranno necessarie due classi \underline{Città} e \underline{Nazione}.\newpage
\section{Considerazioni}
Un \underline{Volo} può appartenere a più di una \underline{Compagnia}? No.\acc 
Un \underline{Volo} può cambiare \underline{Compagnia}? Se si, si vuole mantenere uno storico? No, non si vuole far si che uno stesso 
\underline{Volo} possa cambiare compagnia mantenendo lo stesso codice, sarà semplicemente un altro  \underline{Volo}, con codice 
diverso e \underline{Compagnia} diversa, ma con il tragitto e durata identici.\acc
Una \underline{Nazione} ha almeno una \underline{Città}, data l'esistenza di nazioni a città unica, come San Marino o Singapore.
\section{UML}
\includegraphics[width=\textwidth]{images/UML.png}


\end{document}

