\documentclass[12pt, letterpaper]{article}
\usepackage{graphicx} % Required for inserting images
\usepackage{hyperref}
\usepackage{listings}
\usepackage{amssymb}
\usepackage{amsmath}
\usepackage[english]{babel}
\usepackage{nicefrac, xfrac}
\usepackage{mathtools}
\newcommand{\acc}{\\\hphantom{}\\}
\usepackage[table,xcdraw]{xcolor}
%PER LE BOXED EQUATION
\definecolor{myblue}{rgb}{.8, .8, 1}
\usepackage{empheq}

\newlength\mytemplen
\newsavebox\mytempbox

\makeatletter
\newcommand\mybluebox{%
    \@ifnextchar[%]
       {\@mybluebox}%
       {\@mybluebox[0pt]}}

\def\@mybluebox[#1]{%
    \@ifnextchar[%]
       {\@@mybluebox[#1]}%
       {\@@mybluebox[#1][0pt]}}

\def\@@mybluebox[#1][#2]#3{
    \sbox\mytempbox{#3}%
    \mytemplen\ht\mytempbox
    \advance\mytemplen #1\relax
    \ht\mytempbox\mytemplen
    \mytemplen\dp\mytempbox
    \advance\mytemplen #2\relax
    \dp\mytempbox\mytemplen
    \colorbox{myblue}{\hspace{1em}\usebox{\mytempbox}\hspace{1em}}}
%FINE
\usepackage[paper=a4paper,left=20mm,right=20mm,bottom=25mm,top=25mm]{geometry}
\renewcommand{\labelenumii}{\arabic{enumi}.\arabic{enumii}}
    \renewcommand{\labelenumiii}{\arabic{enumi}.\arabic{enumii}.\arabic{enumiii}}
    \renewcommand{\labelenumiv}{\arabic{enumi}.\arabic{enumii}.\arabic{enumiii}.\arabic{enumiv}}
\title{(FOL) Dipendenti (gruppo 42)}
% \author{ Giacomo Biribicchi \and Marco Casu \and Christian Di Manno \and Alessandro Gautieri }
\date{}
\begin{document}
\maketitle\hphantom{a}\\
Simboli di predicato : $$
\begin{aligned}\mathcal{P}=\{Persona/1,\;CellPersonale/2,\;Numero/1,\;Nome/2\}\;\\
     \cup\; \{Stringa/1,\;Dipartimento/1,\;Lavora/2,\;Direttore/2\}
\end{aligned}$$
Variabili : $$\mathcal{V}=\{x,\;y,\;z,\;a,\;b,\;c\}$$

\textbf{1) Tutte le persone hanno almeno un numero di telefono} \acc 
Con la seguente affermo che ogni $Persona$ ha almeno un $CellPersonale$, che è un $Numero$
\begin{empheq}[box={\mybluebox[2pt]}]{equation*}
    \forall x\; Persona(x) \rightarrow \exists y\; CellPersonale(x,y) \land Numero(y)
\end{empheq}
L'\textbf{Interpretazione} è la seguente: 
$$ \mathcal{D}=\{\alpha,\; \beta, \;\gamma,\; \delta\}$$
$Persona/1 =\{\alpha,\;\beta\}$ \\ 
$Numero/1=\{\gamma,\;\delta\}$\\ 
$CellPersonale/2=\{(\alpha,\gamma),\;(\alpha,\delta)\}$\acc

\textbf{2) Ogni persona ha esattamente un nome}\acc 
Con la seguente affermo che ogni $Persona$ ha almeno un $Nome$, che è una $Stringa$
\begin{empheq}[box={\mybluebox[2pt]}]{equation*}
    \forall x\; Persona(x) \rightarrow \exists y\; Nome(x,y) \land Stringa(y)
\end{empheq}
Con la seguente affermo che non esiste una $Persona$ che ha due $Nomi$ o più
\begin{empheq}[box={\mybluebox[2pt]}]{equation*}
    \lnot \exists \;x,\;y,\;z\; Nome(x,z) \land Nome(x,y) \land  z\ne y
\end{empheq}
L'\textbf{Interpretazione} è la seguente: 
$$ \mathcal{D}=\{\alpha,\; \beta,\; \gamma,\; \delta\}$$
$Persona/1 =\{\alpha,\;\beta\}$ \\ 
$Stringa/1 = \{\gamma,\;\delta\}$\\ 
$Nome/2 = \{(\alpha,\gamma),\;(\beta,\delta)\}$\acc 

\textbf{3) Non ci sono dipendenti che lavorano in più di due dipartimenti}\acc
Con la seguente dico che gli oggetti implicati nella relazione $Lavora$ sono una $Persona$ ed un $Dipartimento$
\begin{empheq}[box={\mybluebox[2pt]}]{equation*}
    \forall x,\;y\; Lavora(x,y)\rightarrow Persona(x) \land Dipartimento(y)
\end{empheq}
Con la seguente dico che, non esiste una persona che lavora in tre (o più) distinti dipartimenti
\begin{empheq}[box={\mybluebox[2pt]}]{equation*}\begin{matrix}
    \lnot \exists\; x,\;a,\;b,\;c \; Persona(x)\; \land \\ 
    Dipartimento(a) \land Dipartimento(b) \land Dipartimento(c)\; \land\\
    Lavora(x,a) \land Lavora(x,b) \land Lavora(x,c) \;\land  a\ne b\ne c
\end{matrix}\end{empheq}
L'\textbf{Interpretazione} è la seguente: 
$$ \mathcal{D}=\{\alpha,\; \beta,\; \gamma,\; \delta,\; \epsilon\}$$
$Persona/1 =\{\alpha,\;\beta\}$ \\ 
$Dipartimento/1=\{\gamma,\; \delta,\; \epsilon\}$\\ 
$Lavora/2=\{(\alpha,\gamma),\;(\alpha,\epsilon),\;(\beta,\gamma),\;(\beta,\delta)\}$\acc

\textbf{4) Ogni dipartimento ha esattamente un direttore che è una persona} \acc 
Con la seguente dico che gli oggetti implicati nella relazione $Direttore$ sono una $Persona$ ed un $Dipartimento$
\begin{empheq}[box={\mybluebox[2pt]}]{equation*}
    \forall x,\;y\; Direttore(x,y)\rightarrow Persona(x) \land Dipartimento(y)
\end{empheq}
Con la seguente affermo che ogni $Dipartimento$ ha almeno un $Direttore$
\begin{empheq}[box={\mybluebox[2pt]}]{equation*}
    \forall y\;Dipartimento(y) \exists x\; Direttore(x,y)
\end{empheq}
Con la seguente affermo che non esiste un $Dipartimento$ con due o più $Direttori$
\begin{empheq}[box={\mybluebox[2pt]}]{equation*}\begin{matrix}
    \lnot \exists\; x,\;a,\;b\; \\ 
    Persona(a) \land Persona(b) \land Dipartimento(x) \land a\ne b \\ 
    Direttore(a,x) \land Direttore(b,x)
\end{matrix}\end{empheq}
L'\textbf{Interpretazione} è la seguente: 
$$ \mathcal{D}=\{\alpha,\; \beta,\; \gamma,\; \delta,\; \epsilon\}$$
$Persona/1 =\{\alpha,\;\beta\}$ \\ 
$Dipartimento/1=\{\gamma,\; \delta,\; \epsilon\}$\\ 
$Direttore/2=\{(\alpha,\gamma),\;(\alpha,\delta),\;(\beta,\epsilon)\}$\acc
\end{document}

