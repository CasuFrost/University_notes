\documentclass[12pt, letterpaper]{article}
\usepackage{graphicx} % Required for inserting images
\usepackage{hyperref}
\usepackage{listings}
\usepackage{amssymb}
\usepackage{amsmath}
\usepackage[english]{babel}
\usepackage{nicefrac, xfrac}
\usepackage{mathtools}
\newcommand{\acc}{\\\hphantom{}\\}
\usepackage[dvipsnames]{xcolor}
\usepackage[table,xcdraw]{xcolor}
\definecolor{light-gray}{gray}{0.95}
\newcommand{\code}[1]{\colorbox{light-gray}{\texttt{#1}}}
\newcommand{\codee}[1]{\colorbox{white}{\texttt{#1}}}
\usepackage[paper=a4paper,left=20mm,right=20mm,bottom=25mm,top=25mm]{geometry}
\renewcommand{\labelenumii}{\arabic{enumi}.\arabic{enumii}}
\renewcommand{\labelenumiii}{\arabic{enumi}.\arabic{enumii}.\arabic{enumiii}}
\renewcommand{\labelenumiv}{\arabic{enumi}.\arabic{enumii}.\arabic{enumiii}.\arabic{enumiv}}
\newcommand{\id}{{\hphantom{ident}}}
\newcommand{\vincolo}[1]{\colorbox{Orange}{$[$\text{#1}$]$}}
\title{\textbf{TuTubi}}

\date{}


\begin{document}

\maketitle\section{Requisiti}
\hphantom{a}\\
1. 

\newpage
\section{Diagramma UML}\begin{center}
    \includegraphics[width=\textwidth ]{images/UML.eps}
\end{center}
\newpage
\section{Diagramma Use-Case}\begin{center}
    \includegraphics[width=\textwidth ]{images/UseCase.eps}
\end{center}
\newpage
\section{Specifiche}
\subsection{Specifica dei tipi di dato}
FileVideo:      \\
\id- sequenza di byte che codifica un flusso video \\    
\id- operazioni del tipo di dato      \\ 
\id durataSec(f:FileVideo): Intero $\ge$ 0\\
\id \id \textit{pre-condizioni} :  nessuna\\
\id \id \textit{post-condizioni} : result è la durata di 'f'\\
\subsection{Specifica delle classi}
\subsubsection{Video}
\code{durataSec () : Intero$\ge$0}\begin{itemize}
    \item \textit{pre-condizioni} : Nessuna
    \item \textit{post-condizioni} : $result$ = durataSec(this.flusso)
\end{itemize}
\code{n\_visualizzazioni () : Intero$\ge$0}\begin{itemize}
    \item \textit{pre-condizioni} : Nessuna
    \item \textit{post-condizioni} : $result$ è il numero \\di oggetti $v:Visualizzazione$ tali che:
    $(this, v): vid_vis$
\end{itemize}
\subsection{Specifica dei vincoli esterni}
\vincolo{V.Video.autore\_non\_risponde\_a\_se\_stesso} : 
Per ogni r:Video, per cui esiste v:Video tale che\\
\id(r,v): risponde\_a\_video\\
\id sia:\\
\id\id- u:Utente tale che:\\
\id\id (u, r): pubblica\\
\id Deve che (u,v) non è un link di pubblica.\\
\id Formalmente:\\
	$$ \begin{matrix}\forall r,v,u \;\;\;  Video(r) \land Video(v) \land Utente(u) \land \\
				pubblica(u,r) \land \\
				risponde\_a\_video(r,v)  \;\;\;\rightarrow\\
                \lnot pubblica(u,v) \end{matrix}$$
                \newpage
\subsection{Specifica degli use-case}
\subsubsection{Cronologia}
\code{cronologia(): Visualizzazione $[$0..*$]$ }\begin{itemize}
    \item \textit{pre-condizioni} : Nessuna
    \item \textit{post-condizioni} : Sia u:Utente l'oggetto che rappresenta l'attore\\ che ha 
    invocato l'operazione.
    $result$ è l'insieme dei\\ $v:Visualizzazione$ tali che:
        $(u,v):ut\_vis$
\end{itemize}
\subsubsection{PubblicazioneVideo}
\code{nuovoVideo(t:Stringa, d:Stringa, f:FileVideo, c:Categoria, T:Tag $[$1..*$]$) : Video }\begin{itemize}
    \item \textit{pre-condizioni} : Nessuna
    \item \textit{post-condizioni} :  
    Viene creato e restituito result:Video tale che:\begin{itemize}
        \item  result.titolo = t
        \item  result.descrizione = d
        \item  result.flusso = f
        \item  result.istante = adesso
\end{itemize}
Viene creato il link $(c, result): cat\_video$.

Per ogni $t:T$, viene creato il link $(t,result): tag\_video$.

Sia $u:Utente$ l'oggetto che rappresenta l'attore che ha 
invocato l'operazione.

Viene creato il link $(u,result): utente\_pubblica\_video$.
\end{itemize}
\subsubsection{Registrazione}
\code{nuovoUtente(nome:Stringa) : Utente}\begin{itemize}
    \item \textit{pre-condizioni} : $\nexists u:Utente$ tale che $u.nome = nome$
    \item \textit{post-condizioni} :  
    viene creato e restituito $result:Utente$ tale che:

				$result.nome$ = nome

				$result.iscrizione=now$ 
\end{itemize}
\end{document}