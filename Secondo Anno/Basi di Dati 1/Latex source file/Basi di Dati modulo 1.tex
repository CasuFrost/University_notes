\documentclass[12pt, letterpaper]{article}
\usepackage{graphicx} % Required for inserting images
\usepackage{hyperref}
\usepackage{xcolor}
\usepackage{amssymb}
\usepackage{amsmath}
\usepackage[english]{babel}
\definecolor{light-gray}{gray}{0.95}
\usepackage[paper=a4paper,left=20mm,right=20mm,bottom=25mm,top=25mm]{geometry}
\newcommand{\code}[1]{\colorbox{light-gray}{\texttt{#1}}}
\title{Basi di Dati 1}
\author{Marco Casu}
\date{\vspace{-5ex}}
\begin{document}



\maketitle
\begin{figure}[h]
    \centering{
    \includegraphics[width=0.7\textwidth ]{images/DatabaseStilizzato.png}
    }
\end{figure}
\newpage 
\section{Introduzione}
L'informazione memorizzata nei sistemi elettronici può essere di due tipi, \textbf{strutturata}
e \textbf{non strutturata}. In questo corso ci occuperemo dell'informazion strutturata, ossia composta da 
oggetti matematici ben definiti. Un \textbf{sistema informativo} connette e contiene le informazioni,
alla quale si può accedere da diversi componenti. Si prenda come esempio di sistema informativo l'archivio
fisico dei documenti (ossia i dati) di una azienda, ad esso, possono accedere i vari reparti, come la sezione 
commercio o le risorse umane, ognuno ha a disposizione l'accesso ad un determinato sotto-insieme di dati (permessi).
Prima di adoperare i sistemi informativi, ogni reparto aveva il suo personale archivio, ciò faceva si che molti dei dati
fossero duplicati e presenti per più reparti, causando un elevata \textbf{ridondanza} di dati. Inoltre, due dati da 2 archivi
diversi potrebbero dipendere tra loro. È quindi importante mantenere l'informazione \textbf{centralizzata}.
I dati vanno organizzati, gestiti, e regolamentati da permessi di accesso a secondo dell'utente che vuole accedervi.
\\\\Un sistema informativo è composto dai seguenti componenti : 
\begin{itemize}
    \item Database (DB)
    \item Database Managment System (DBSM) - Ossia il software per il mantenimento dei dati  
    \item Application Software
    \item Computer Hardware - La memoria nella quale è contenuto
\end{itemize}
È importante mantenere una visione astratta del modello di sistema informativo, che sia indipendente
dall'hardware in uso. Fissiamo un modello che utilizzi tipi di dati che non possano variare nel tempo,
strutturati in maniera completa, è necessario pensare ad un metodo formale per organizzare i dati.
Vogliamo farlo in un ambiente condiviso, preoccupandoci quindi di quali e quanti dati vengono condivisi condivisi
con i singoli utenti, che avranno permessi diversi e potranno accedere a dati diversi (ad esempio, uno studente può
accedere ai suoi esami registrati, ma non a quelli degl altri). Inolte bisogna anche amministrare i metodi
con cui si accede ai dati, preoccupandoci della concorrenza (accedere allo stesso dato nello stesso momento).
I dati vengono organizzati in maniera omogenea, esiste un tipo di dato per descrivere un entità (ad esempio uno studente),
e tutte le volte che si vuole immagazinare nel sistema uno studente, bisogna ricorrere allo stesso tipo di dato.
L'informazione viene rappresentata da un aggregamento di più dati \textit{grezzi}, ossia, rispettivamente la stringa
 "\textit{Maurizio Ernesti}" e l'intero "3761523746", insieme rappresentano l'informazione di un 
 professore (nome, cognome e numero di telefono).\\\\
 Il modello è il modo in cui decidiamo di organizzare e collegare i dati, esistono modelli \textbf{logici}, 
 indipendenti dalla struttura fisica, come il \textbf{modello relazionale}, ed esistono modelli \textbf{concettuali},
 ossia rappresentazioni ancora più astratte indipendenti dal modello logico, i primi modelli sono stati introdotti
 negli anni 60.
 \\\\
 \begin{minipage}{0.16\textwidth}
    \includegraphics[width=\linewidth]{images/GrafoViola.png}
    \end{minipage}
    \begin{minipage}{0.8\textwidth}\raggedright
        un esempio è il modello \textit{Mesh}, rappresentato con un grafo, dove i nodi sono i dati (i record), e
        gli archi le loro relazioni.
    \end{minipage}
    \noindent
    \\\\
In questo modello i collegamenti sono esplicitati fisicamente (possiamo immaginare con dei puntatori), i modelli
relazionali, diversamente, hanno relazioni rappresentate implicitamente dai valori stessi che contengono.
\\

\centering
\begin{tabular}{|l|l|l|r|}
    \hline
\multicolumn{4}{|c|}{\textbf{Studenti}}\\
    \hline
    Matricola & Cognome & Nome & Compleanno\\
    \hline
    276545 & Gialli & Lucia & 25/11/1980\\
    \hline
    176515 & Rossi & Mario & 13/09/1982\\
    \hline
    348191 & Verdi & Andrea & 04/07/1981\\
    \hline
    \end{tabular}
\begin{tabular}{|l|l|l|r|}
    \hline
\multicolumn{3}{|c|}{\textbf{Esami}}\\
    \hline
    Studente & Voto & Corso \\
    \hline
    348191 & 25 & 01 \\
    \hline
    176515 & 18 & 02 \\
    \hline
    176515 & 27 & 02 \\
    \hline
    \end{tabular}
\\ 
 \hphantom{.}\\
 \raggedright
La relazione tra gli esami e gli studenti (ogni studente ha \(n\) esami registrati) è data implicitamente
dalla presenza del campo \textit{Studente} nella tabella \textit{Esami}, che equivale al campo \textit{Matricola}
della tabella \textit{Studenti}, il modello relazionale è basato su oggetti, classi ed attributi.
Ad esempio, se volessi sapere che voto ha preso \textit{Mario Rossi} al corso numero 02, mi basterebbe
consultare la tabella \textit{Esame}, e controllare il campo che ha la matricola equivalente a quella di \textit{Mario Rossi}.
\\Lo schema logico descrive la presenza di tutte le entità con i loro rispettivi attributi, esistono poi gli schemi
esterni, ossia sotto-insiemi degli schemi, destinati a determinati tipi di utenti che ne hanno l'accesso. Anche lo schema 
logico completo può essere schema esterno, il super-amministratore di un sistema informativo, come schema esterno, avrà 
l'intero schema logico, avendo accesso a tutte le entità.
\\Come già detto, lo schema logico rappresenta la struttura degli oggetti/entità, ed è invariata nel tempo, ma la 
sua istanza, ossia gli effettivi campi delle tabelle, possono variare, ed anche rapidamente. Per gestire gli schemi si utilizzano
dei veri e propri linguaggi.
\begin{itemize}
    \item \textbf{Data Definition Language (DDL)} - Per la definizione degli schemi logici ed altre operazioni generali.
    \item \textbf{Data Manipulation Language (DML)} - Per interrogare lo schema logico, leggerne i valori ed eventualmente modificarli. 
\end{itemize}
È largamente utilizzato il linguaggio \textbf{SQL (Structured Query Language)}, che funge sia da DDL che da DML.
\\Un base di dati deve essere :
\begin{itemize}
    \item Manipolabile
    \item Modificabile
    \item Centralizzata
    \item Il minimo ridondante
    \item Sicura
\end{itemize}
 I dati molto spesso devono soddisfare certi vincoli (ad esempio, ogni studente ha una sola residenza)
 , tali vincoli son chiamati \textbf{dipendenze funzionali}, e possono riguardare anche il dominio di certi
 attributi (ad esempio, il voto verbalizzato di un esame deve essere maggiore o uguale a 18 e minore o uguale a 30).
 I dati devono essere protetti da accessi non autorizzati, è necessaria la dichiarazione di regole di accesso.
\\Definiamo adesso uno specifico tipo di operazione sulle basi di dati, ossia le \textbf{transizioni}, che non sono altr
che sequenze ordinate di operazioni che vanno obbligatoriamente eseguite insieme in sequenza, con il divieto
assoluto che ne venga eseguita solo una parte.
Facciamo un esempio, si dia il caso che su una base di dati bancaria, si vogliano trasferire 1000 euro dal conto \textit{C1} 
al conto \textit{C2}, le seguenti operazioni definite informalmente sono :
\begin{itemize}
    \item Cercare il conto \textit{C1}
    \item Sottrarne al bilancio 1000
    \item Cercare il conto \textit{C2}
    \item Aggiungerne al bilancio 1000
\end{itemize} 
Se per errore si eseguono solo i primi 2 passi, ci si ritroveranno 1000 euro persi, sottratti al primo conto, ma non
addizionati al secondo. Una transizione va quindi completamente eseguita, se non dovesse essere così, l'intera sequenza
di operazioni va abortita. Inoltre, è importante notare che data la concorrenzialità, in una base di dati potrebbe
accadere di accedere allo stesso dato nello stesso momento, ciò potrebbe causare errori, è quindi importante evitare
di lavorare contemporaneamente su uno stesso specifico campo.
\section{Il Modello Relazionale}
Il modello relazionale è basato sulla relazione intesa in senso matematico, una relazione non è altro che un
insieme di tuple, tutte della stessa lunghezza, con elementi appartenenti a diversi domini. Il dominio è 
l'insieme dei possibili valori che gli elementi delle tuple possono assumere. Se prendiamo una lista di \(k\) domini,
il prodotto cartesiano di tutti i \(k\) domini è l'insieme di tuple di lunghezza \(k\).
\begin{equation}
    D_1 \times D_2 \times D_3....\times D_k = \{(v_1,v_2,v_3...,v_k)|v_1\in D_1,v_2\in D_2,v_3\in D_3...,v_k\in D_k\}
\end{equation}
Per esempio, per \(k=2\) consideriamo i seguenti domini \(D_1=\{White,Black\}\) e \(D_2=\{0,1,2\}\), si ha che :
\begin{equation}
    D_1\times D_2 = \{(White,0),(White,1),(White,2),(Black,0),(Black,1),(Black,2)\}
\end{equation}
È una relazione di grado 2, perchè ogni tupla ha 2 coordinate.
\newtheorem{theorem}{Teorema}
\begin{theorem}
    Il grado di una relazione è equivalente al numero di elementi di ogni tupla appartenente a tale relazione.
    \end{theorem}
In tale prodotto cartesiano è possibile costruire \(2^6\) possibili relazioni, ossia sotto-insiemi del
prodotto cartesiano.
\begin{theorem}
    Una relazione è un qualsiasi sottoinsieme del prodotto cartesiano.
\end{theorem}
\subsection{Notazione con Indice}\label{notConInd}
Sia \(r\) una relazione di grado \(k\), data \(t\) una tupla appartenente alla 
relazione \(r\), ed \(i\) un intero da \(1\) a \(k\), con \(t[i]\) si intende l'elemento
alla cooridnata \(i\)-esima della tupla \(t\).
\\
\hphantom{.}\\
\centering
\begin{tabular}{|l|l|l|r|}
    \hline
\multicolumn{2}{|c|}{\textbf{Tabella}}\\
    \hline
    Black & 0 \\
    \hline
    White & 0 \\
    \hline
    \end{tabular}\\
    \raggedright
Se prendiamo \(t\) come la prima riga della tabella, si avrà che \(t[1]=\)
"\textit{Black}" e  \(t[2]=0\).\\
Ricordando la notazione tabellare, in basi di dati le intestazioni delle colonne ed il loro dominio
è denominato \textbf{attributo} (ad Esempio \textit{Name : string}).
In una tabella, due attributi distinti non possono avere lo stesso nome.\\
\subsection{Rappresentazione come Funzioni}
Sia \(R\) un oggetto definito come insieme di attributi, una tupla su \(R\), ossia un istanza di
tale oggetto, può essere visto come una funzione definita su \(R\) che associa ad ogni attributo \(A\), un valore
presente nel dominio di \(A\). Considerando ciò, presa \(t\) una tupla di \(R\), ed \(A\) uno dei suoi attributi,
indichiamo con \(t(A)\), il valore (ossia l'istanza) di quell'attributo preso dalla funzione \(t\) sulla variabile \(A\).

Ad esempio, la relazione \(R\) ha la tupla \(t_1=(Paolo,Rossi,2,26.5)\), considerando \(t_1\) come una funzione,
essa associa ad ogni attributo \(A\), un elemento del suo dominio :
\begin{equation}
    f : (Nome,Cognome,Esami,Media)\rightarrow string\cup string\cup int \cup real
\end{equation}
Da qui si ha \hphantom{aa} \(t_1(Nome)=Paolo\)\hphantom{aa}  \(t_1(Cognome)=Rossi\)\hphantom{aa} \(t_1(Media)=26.5\)\\


Lo schema logico, è il dominio di tale funzione, l'istanza sono le tuple, le istanze sono insiemi di tuple,
ossia una relazione.
\begin{theorem}
    L'istanza è un sotto-insieme di tuple
\end{theorem}
Ogni riga della tabella è una tupla distinta, ed ogni colonna corrisponde al dominio. Possiamo quindi rappresentare
un oggetto della base di dati con la seguente notazione : 
\begin{equation}
    R(A_1,A_2,A_3...,A_k)
\end{equation}
Qui \(R\) è una relazione dello schema, quindi uno schema di basi di dati non è altro che un 
insieme di relazioni \((R_1,R_2,R_3...,R_k)\) (invarianti nel tempo) per le quali, ognuna di esse possiede un istanza (variante nel tempo).
Da qui in poi, utilizzeremo \(R\) per denominare le relazioni, ed \(r\) per le loro istanze.\\
Riprendendo la notazione con indice vista in precedenza \ref{notConInd}, si pò utilizzare puttosto
che un indice intero, l'intestazione dell'attributo per il quale si voglia leggere l'istanza.\\
\centering
\hphantom{.}\\
\begin{tabular}{|l|l|l|r|}
    \hline
\multicolumn{3}{|c|}{\textbf{Luoghi}}\\
    \hline
    Città & Regione & Popolozione \\
    \hline
    Roma & Lazio & 3000000 \\
    \hline
    Milano & Lombardia & 1500000 \\
    \hline
    Genova & Liguria & 150000 \\
    \hline
    \end{tabular}
\\ 
 \hphantom{.}\\
 \raggedright
 Data \(t_1\) la prima riga dell'istanza, si ha \(t_1[\)Città\(]\)="\textit{Roma}". È possibile anche farlo
 con sottoinsiemi di attributi, ossia \(t_1[\)Regione,Popolazione\(]\)=("\textit{Lazio}",\(3000000\)). Non è importante
 l'ordine degli attributi come argomenti, come "risultato" riceviamo una sotto-tupla della tupla \(t_1\), detta \textit{restrizione}.
Abbiamo visto come le istanze delle tabelle non sono altro che insiemi di tuple, esistono tante possibili istanze quanto la cardinalità
del prodotto delle cardinalità dei domini.\\
Può succedere in certi casi, che nell'istanza di una relazione, sia presente una riga in cui un attributo è \textbf{sconosciuto},
per rappresentare tale campo nelle basi di dati, si utilizzi il valore polimorfico\footnote{Appartente a tutti i domini} \(NULL\), utilizzato
per riempire gli spazi vuoti, ad esempio in una tabella contente i dati degli utenti iscritti ad un sito, è possibile che alcuni utenti abbiano
omesso il numero di telefono, per loro il campo avrà valore \(NULL\), ossia sconosciuto. Si ricordi che \(NULL\) è diverso da \(0\).
\\\centering
\hphantom{.}\\
\begin{tabular}{|l|l|l|r|}
    \hline
\multicolumn{3}{|c|}{\textbf{Luoghi}}\\
    \hline
    Matricola & Nome & Cellulare \\
    \hline
    1039 & Luca & 3475746371 \\
    \hline
    4316 & Giorgio & \(NULL\) \\
    \hline
    1499 & Sandro & 3857482845 \\
    \hline
    \end{tabular}\\
 \raggedright
 \subsection{Integrità dei Dati}\label{intDeiDati}
 La presenza di un valore \(NULL\) può causare alcuni errori, vedremo come sono presenti alcuni attributi, le quali
 istanze devono per forza essere dichiarate e non sconosciute. Esistono però diversi tipi di errori, come
 dei campi identificativi duplicati o valori fuori dominio.
 \\\centering
\hphantom{.}\\
\begin{tabular}{|l|l|l|r|}
    \hline
\multicolumn{3}{|c|}{\textbf{Studenti}}\\
\hline
Matricola & Nome & Media \\
\hline
\color{red}1039\color{black} & Luca & 28 \\
\hline
4316 & Giorgio & \color{blue}33\color{black} \\
\hline
\color{red}1039\color{black} & Sandro & 22 \\
\hline
\end{tabular}
\\ 
 \hphantom{.}\\
 \raggedright
 \begin{itemize}
    \color{red}\item Errore 1 - \color{black} Nella prima e nella terza riga sono presenti due studenti con la stessa matricola. Il campo matricola 
    identifica ogni singolo e distinto studente, e non può essere duplicato.
    \color{blue}\item Errore 2 - \color{black} Uno studente ha come media dei voti 33, è impossibile dato che i voti sono compresi tra
    18 e 30, è quindi un valore fuori dominio.
    
 \end{itemize}
Tali errori vengono definiti problemi di \textbf{integrità dei dati}, per mantenere tale integrità è necessario
che le istanze delle relazioni soddisfino delle determinate proprietà dette \textbf{vincoli}. Vedremo che esistono :
\begin{itemize}
    \item Vincoli di chiave
    \item Vincoli di dominio
    \item Vincoli funzionali
    \item Vincoli di esistenza
\end{itemize}
\centering
\hphantom{.}\\
\begin{tabular}{|l|l|l|l|l|l|r|}
    \hline
\multicolumn{6}{|c|}{\textbf{Impiegati}}\\
\hline
Codice Impiegato & Nome & Cognome & Ruolo & Assunzione & Dipartimento\\
\hline
01 & Luca & Rossi & Analista & \color{red}1785 & 01 \\
\hline
\color{red}02 & Mario & Verdi & Amministratore & 1980 & 02 \\
\hline
\color{red}02 & Giorgio & Neri & Ricercatore & 1985 & \color{red}05 \\
\hline
\end{tabular}\\\hphantom{.}\\\hphantom{.}\\
\begin{tabular}{|l|l|r|}
    \hline
\multicolumn{2}{|c|}{\textbf{Dipartimenti}}\\
\hline
Numero & Nome \\
\hline
01 & Managment \\
\hline
02 & Amministrazione \\
\hline
\end{tabular}\\
 \hphantom{.}\\
 \raggedright
Vediamo come nello schema logico appena mostrato ci sono diversi vincoli da definire che non sono rispettati.
Ad esempio, va definito il vincolo di dominio per cui il valore "Assunzione"\(\ge1980\), ossia tale valore deve essere
strettamente maggiore di una certa data (possibile data di nascita dell'azienda). Si noti che non si sta rispettando
un vincolo di chiave, dato che il "Codice Impiegato" presente nella seconda e nella terza riga lo stesso valore, essendo
esso l'attributo identificativo, non deve essere duplicato ("Codice Impiegato" \(UNIQUE\)). Un altro errore meno evidente, è che
alla terza riga della tabella "Impiegati", è presente un campo dipartimento con codice \(05\), tale campo dovrebbe
collegare quella riga con il suo rispettivo dipartimento presente su un altra tabella, ma notiamo che nella tabella "Dipartimenti", 
non è presente alcuna riga con "Numero" identificativo \(05\). Un altro vincolo noto è il vincolo di esistenza, che impone ad un certo
attributo di non accettare valore \(NULL\).\\\hphantom{.}\\
Un altra importante distinzione da fare tra vincoli è di suddividerli in :
\begin{itemize}
    \item \textbf{Intra-relazionali} - Vincoli definiti e da soddisfare all'interno della singola tabella (Ad esempio, un vincolo di dominio 
    per il quale un valore deve essere sufficentemente grande).
    \item \textbf{Inter-relazionali} - Vincoli soddisfatti dai collegamenti di più tabelle (Un chiaro esempio è il sopra-citato errore sulla tabella del 
    dipartimento).
\end{itemize}
\subsection{Le Chiavi}
In un istanza \(r\) di una relazione \(R\), per ogni tupla è necessario che vi sia un attributo (o un insieme di attributi ) \(X\)
che la identifichi e distingua dalle altre tuple (che quindi non sia mai duplicato). Tale attributo/insieme di attributi è detto
\textbf{chiave}, un chiaro esempio può essere il campo "Matricola" all'interno di un'ipotetica tabella studenti.
Vediamo una definizione più formale.
\begin{theorem}
    \hphantom{.}\\\textbf{ Punto 1 }- Per ogni istanza della relazione \(R\), non esistono due tuple \(t_1,t_2\) che hanno gli stessi valori
    per tutti i singoli attributi, preso un insieme di attribtui \(X\), vale sempre \(t_1[X]\ne t_2[X]\).
    \begin{equation}
        \text{sia }X\in R(A_1,A_2...,A_k) | \forall t_1,t_2 \in r \text{ se }t_1[X]=t_2[X] \implies t_1=t_2
    \end{equation}
    \textbf{ Punto 2 }- Inoltre, non esistono sotto-insiemi di \(X\) che soddisfino la condizione sopra-citata.
    \begin{equation}
        \forall X'\subseteq X, \exists t_1,t_2 \in r \text{ tale che } t_1[X']=t_2[X'] \land t_1 \ne t_2
    \end{equation}
\end{theorem}
Approfondendo il punto 2, se esiste una relazione \(R\) con chiave \(X=(A_1,A_2,A_3)\), è chiaro che, nella 
sua istanza \(r\), non esisteranno due righe con gli stessi valori assegnati \(X\), quindi, preso il sotto-insieme
\(X'=(A_1,A_2)\), se esistono due tuple \(t_1,t_2\) tali che \(t_1[X']=t_2[X']\), per forza di cose esse non 
saranno la stessa tupla in quanto, per il punto 1, differiranno per l'attributo \(A_3\).
È importante per una relazione che abbia una chiave significativa basata sull'informazione che rappresenta
( Una relazione che rappresenta degli studenti, non può avere come chiave il campo del nome, dato che potrebbero 
esserci 2 studenti con lo stesso nome, bensì si prediliga la matricola), si sceglie quindi una \textbf{chiave primaria},
ovviamente con vincolo di esistenza.
\subsubsection{La Chiave Esterna}
Può esistere inoltre un attributto nelle relazioni detto \textbf{chiave esterna} o \textbf{foreign key},
essa identifica all'interno di una relazione, un attributo associato ad un altra relazione (un altra tabella),
identificandone la chiave primaria.
\\ \hphantom{.}\\
\centering
\begin{tabular}{|l|l|l|r|}
    \hline
\multicolumn{4}{|c|}{\textbf{Studenti}}\\
    \hline
    \color{blue}Matricola & Cognome & Nome & Compleanno\\
    \hline
    276545 & Gialli & Lucia & 25/11/1980\\
    \hline
    176515 & Rossi & Mario & 13/09/1982\\
    \hline
    348191 & Verdi & Andrea & 04/07/1981\\
    \hline
    \end{tabular}
\begin{tabular}{|l|l|l|r|}
    \hline
\multicolumn{3}{|c|}{\textbf{Esami}}\\
    \hline
    \color{blue}Studente & Voto & Corso \\
    \hline
    348191 & 25 & 01 \\
    \hline
    176515 & 18 & 02 \\
    \hline
    176515 & 27 & 02 \\
    \hline
    \end{tabular}
\\ 
 \hphantom{.}\\
 \raggedright
Nella tabella "\textit{Esami}", l'attributo "\textit{Studente}" è una chiave esterna che identifica e collega 
tale tabella con la relazione "\textit{Studenti}", tramite la sua chiave primaria "\textit{Matricola}",
Dato che ad ogni esame è associato uno studente che l'ha sostenuto. Il vincolo di integrità inter-relazionale
precedentemente citato impone che per ogni valore presente su un attributo di una chiave esterna, esista 
il suo corrispettivo campo con tale valore nella relazione alla quale fa riferimento. Si ricordi che tale vincolo
non è violato dalla presenza di un valore \(NULL\). 
\\ \hphantom{.}\\
\centering
\begin{tabular}{|l|l|l|r|}
    \hline
\multicolumn{4}{|c|}{\textbf{Multe}}\\
    \hline
    Codice & Data & Ufficiale & Targa\\
    \hline
    4312 & 01/12/1988 & 001 & AA123AA\\
    \hline
    1351 & 12/04/1989 & \color{blue}\(NULL\) & KA194AR\\
    \hline
    9572 & 10/10/1990 & 002 & ND193MF\\
    \hline
    \end{tabular}
\begin{tabular}{|l|l|l|r|}
    \hline
\multicolumn{3}{|c|}{\textbf{Ufficiali}}\\
    \hline
    Codice & Nome & Cognome \\
    \hline
    001 & Giancarlo & Pozzi \\
    \hline
    002 & Sara & Tua \\
    \hline
    003 & Nicola & Canti \\
    \hline
    \end{tabular}
\\ 
 \hphantom{.}\\
 \raggedright
Può quindi esistere un istanza dove la chiave esterna ha valore \(NULL\),
conseguentemente non avrà nessun riferimento nella relazione alla quale è collegata. 
\subsection{Le Dipendenze Funzionali}
Come si possono definire facilmente i vincoli di integrità dei dati\ref{intDeiDati} visti nei paragrafi 
precedenti? Nelle basi di dati si utilizzano le note \textbf{dipendenze funzionali}, ossia un insieme
di attributi che dipende da un altro insieme di attributi all'interno dello stesso schema. Tale definizione
può sembrare poco esplicativa, ma è di vitale importanza in questo paragrafo che lo studente abbia
ben saldo in mente il concetto di dipendenza funzionale, in quanto centrale nel corso di \textit{Basi di Dati}.
Formalmente, una dipendenza funzionale stabilisce un collegamento semantico tra due distinti insiemi 
di attributi \(X\) e \(Y\) appartenenti allo stesso schema. Si scrive:
\begin{equation}
    X\rightarrow Y
\end{equation}
e si legge : "X \textbf{determina} Y", stabilendone dei vincoli di integrità.
\begin{theorem}
    Sia \(r\) un istanza della relazione \(R\), la dipendenza funzionale \(X\rightarrow Y\) è soddisfatta se :
    \begin{itemize}
        \item Sia \(X\) che \(Y\) sono due sotto-insieme distinti di \(R\).
        \item Le tuple di \(r\) che sono identiche per \(X\), sono anche identiche per \(Y\). \begin{equation}
            \forall t_1,t_2 \in R \text{ se } t_1[X]=t_2[X] \implies t_1[Y]=t_2[Y]
        \end{equation}
    \end{itemize}
\end{theorem}
\newpage\section{Algebra relazionale}
Tramite \textbf{l'algebra relazionale} possiamo fare interrogazioni alla nostra base di dati per 
ottenere informazioni su una porzione di essa, le \textit{query} sono scritte in linguaggio SQL, e vengono
tradotte nel linguaggio formale e procedurale dell'algera relazionale. L'algebra relazionale fornisce 
degli operatori che lavorano sulle istanze delle nostre relazioni, vi sono 4 tipi di operatori :
\begin{itemize}
    \item Operatori di rimozione sulle singole relazioni
    \item Operatori insiemistici 
    \item Operatori che combinano tuple da relazioni diverse 
    \item Operatore di rinomina
\end{itemize}
Vediamo nel dettaglio tutti gli operatori che abbiamo a disposizione :
\subsubsection{Proiezione}
L'operatore di proiezione, indicato con \(\pi\), esegue un \textit{taglio verticale} su una 
tabella, selezionando un sotto-insieme degli attributi, creando una tabella come quella iniziale, 
ma con esclusivamente le istanze degli attributi selezionati.\begin{center}
    \( \pi_{A_1,A_2,..A_k}(R) \) 
\end{center}
\textit{Esempio :}\begin{center} Sia : 
    \begin{tabular}{|l|l|l|r|}
        \hline
    \multicolumn{3}{|c|}{\textbf{Customer}}\\
        \hline
        Code & Name & Town \\
        \hline
        001 & Giancarlo & Roma \\
        \hline
        002 & Sara & Cagliari \\
        \hline
        003 & Nicola & Roma \\
        \hline
        \end{tabular}
        ho che \(\pi_{Name}(Customer)=\)
        \begin{tabular}{ |l|r| } 
            
            \hline Roma \\ 
            \hline Cagliari  \\ 
            \hline
           \end{tabular}
\end{center}
\subsubsection{Selezione}
La selezione, indicata con \(\sigma\), esegue un \textit{taglio orizzontale} sulla tabella, ossia, seleziona tutte le righe 
che soddisfano un determinato vincolo \(C\), tale vincolo è un sepressione booleana della forma 
\(A\Theta B\), dove \(\Theta\in\{<,>,=,\le,ge\}\). La condizione ovviamente per esser valida, ha bisogno 
che \(A\) e \(B\) abbiano lo stesso dominio (non posso comparare un intero con una stringa).\begin{center}
    \( \sigma_{C}(R)\)
\end{center}
\textit{Esempio :}\begin{center} Sia : 
    \begin{tabular}{|l|l|l|r|}
        \hline
    \multicolumn{3}{|c|}{\textbf{Customer}}\\
        \hline
        Code & Name & Town \\
        \hline
        001 & Giancarlo & Roma \\
        \hline
        002 & Sara & Cagliari \\
        \hline
        003 & Nicola & Roma \\
        \hline
        \end{tabular}
        ho che :\end{center}\begin{center}  \( \sigma_{Town="Roma"}(Customer)=\)
        \begin{tabular}{|l|l|l|r|}
            \hline
   
            Code & Name & Town \\
            \hline
            001 & Giancarlo & Roma \\
            \hline
            003 & Nicola & Roma \\
            \hline
            \end{tabular}
\end{center}
Ovviamente posso utilizzare gli operatori logici per eseguire richieste multiple :
\begin{center}  \( \sigma_{Town="Roma"\land Code=001}(Customer)=\)
    \begin{tabular}{|l|l|l|r|}
        
        \hline
        Code & Name & Town \\
        \hline
        001 & Giancarlo & Roma \\
        \hline
        \end{tabular}
        \\\hphantom{.}\\Vale la proprietà : \(\sigma_{C}(\sigma_{C'}(R))=\sigma_{C\land C'}(R)\)
    \end{center}
    \subsubsection{Unione}
L'operazione di unione fra due relazioni, denominata con il simbolo \(\cup\), come nel senso insiemistico, crea una nuova istanza 
della relazione, contenente le tuple da entrambe le relazioni. Non si possono unire due relazioni 
qualsiasi, è necessario che esse siano \textbf{union-compatibili}, ossia che abbiano lo stesso 
numero di attributi, e che gli attributi corrispondenti abbiano lo stesso dominio.
\begin{center}
    Siano : 
    \begin{tabular}{|l|l|r|}
        \hline
    \multicolumn{2}{|c|}{\textbf{Insegnanti}}\\
        \hline
        Code & Name  \\
        \hline
        \color{blue}001 & \color{blue}Giancarlo  \\
        \hline
        002 & Sara  \\
        \hline
        003 & Nicola\\
        \hline
        \end{tabular}
        \begin{tabular}{|l|l|r|}
            \hline
        \multicolumn{2}{|c|}{\textbf{Admin}}\\
            \hline
            Code & Name  \\
            \hline
            \color{red}001 & \color{red}Luca  \\
            \hline
            004 & Andrea  \\
            \hline
            005 & Carlo\\
            \hline
            \end{tabular}
 Si ha che :\end{center}\begin{center}
\(Insegnanti\cup Admin = \)
\begin{tabular}{|l|l|r|}
    \hline
    Code & Name  \\
    \hline
    \color{blue}001 & \color{blue}Giancarlo  \\
    \hline
    002 & Sara  \\
    \hline
    003 & Nicola\\
    \hline
    004 & Andrea  \\
    \hline
    005 & Carlo\\
    \hline
    \end{tabular} \end{center}
    Si noti come la riga 
    \begin{tabular}{|l|l|r|}
        \hline
        \color{red}001 & \color{red}Luca \\  \hline
    \end{tabular} 
non appare nell'unione, dato che condivide la stessa chiave con la riga 
\begin{tabular}{|l|l|r|}
    \hline
    \color{blue}001 & \color{blue}Giancarlo \\  \hline
\end{tabular} 
, quindi, quando si uniscono due relazioni, bisogna fare attenzione che due righe da esse non condividano 
la stessa chiave, altrimenti si avrà una perdita di informazioni. Si noti che, se una delle due tabelle avesse 
presentato un attributo di troppo, esso sarebbe potuto essere stato rimosso con una proiezione, rendendole comunque 
union-compatibili.
\subsubsection{Differenza}
Come per l'unione, è necessario che le due relazioni sulla quale voglio applicare la differenza siano 
union-compatibili. Si indica con il simbolo \(-\), ed il risultato fra due relazioni, conterrà 
le tuple del primo operando, che non stanno anche nel secondo operando.
\begin{center}
    Siano : 
    \begin{tabular}{|l|l|r|}
        \hline
    \multicolumn{2}{|c|}{\textbf{Insegnanti}}\\
        \hline
        Code & Name  \\
        \hline
        001 & Giancarlo  \\
        \hline
        002 & Sara  \\
        \hline
        \end{tabular}
        \begin{tabular}{|l|l|r|}
            \hline
        \multicolumn{2}{|c|}{\textbf{Admin}}\\
            \hline
            Code & Name  \\
            \hline
            002 & Sara  \\
            \hline
            \end{tabular}
 Si ha che :\end{center}\begin{center}
    \(Insegnanti-Admin =\)
    \begin{tabular}{|l|l|r|}
        \hline
        Code & Name  \\
        \hline
       
        001 & Giancarlo  \\
        \hline
        \end{tabular}
 \end{center}
 La differenza non è commutativa! Infatti, in questo caso \(Admin-Insegnanti=\emptyset \)
\newpage
\subsubsection{Intersezione}
Anch'essa richiede che i due operandi siano union-compatibili. Si indica con \(\cap\), ed il
risultato conterrà esclusivamente le tuple che fanno parte sia della prima che della seconda relazione.
(si noti che \(A\cap B = A - (A-B)\)).
\begin{center}
    Siano : 
    \begin{tabular}{|l|l|r|}
        \hline
    \multicolumn{2}{|c|}{\textbf{Insegnanti}}\\
        \hline
        Code & Name  \\
        \hline
        001 & Giancarlo  \\
        \hline
        002 & Sara  \\
        \hline
        \end{tabular}
        \begin{tabular}{|l|l|r|}
            \hline
        \multicolumn{2}{|c|}{\textbf{Admin}}\\
            \hline
            Code & Name  \\
            \hline
            002 & Sara  \\
            \hline
            004 & Andrea  \\
            \hline
            \end{tabular}
 Si ha che :\end{center}\begin{center}
    \(Insegnanti\cap Admin =\)
    \begin{tabular}{|l|l|r|}
        \hline
        Code & Name  \\
        \hline
        002 & Sara  \\
        \hline
        \end{tabular}
 \end{center}
 Vediamo adesso gli operatori della terza categoria, che creano \textbf{relazioni multiple}
 combinando tabelle eterogenee.
 \subsubsection{Prodotto Cartesiano}
 Il prodotto cartesiano è indicato con il simbolo \(\times\), e rappresenta tutte le possibili 
 combinazioni di righe fra gli elementi del primo operando con gli elementi del secondo, tale 
 operazione risulta parecchio dispendiosa, e potrebbe associare elementi che non hanno 
 nessuna correlazione sensata, ad esempio :
 \begin{center}
    Siano : 
    \begin{tabular}{|l|l|r|}
        \hline
    \multicolumn{2}{|c|}{\textbf{Customer}}\\
        \hline
        Code & Name  \\
        \hline
        001 & Giancarlo  \\
        \hline
        002 & Sara  \\
        \hline
        \end{tabular}
        \begin{tabular}{|l|l|r|}
            \hline
        \multicolumn{2}{|c|}{\textbf{Order}}\\
            \hline
            CO & Customer  \\
            \hline
            AX00 & 001  \\
            \hline
            AX01 & 002  \\
            \hline
            \end{tabular}
 Si ha che :\end{center}\begin{center}
    \(Customer\times Order =\)
    \begin{tabular}{|l|l|l|l|r|}
        \hline
        Code & Name & CO & Customer  \\
        \hline
        001 & Giancarlo & AX00 & 001 \\\hline
        001 & Giancarlo & AX01 & 002 \\\hline
        002 & Sara & AX00 & 001 \\\hline
        002 & Sara & AX01 & 002 \\
        \hline
        \end{tabular}
 \end{center}
 Logicamente però, stiamo associando a dei clienti, degli ordini che non gli appartengono (il campo \(Customer\) 
 di \(Order\) è la foreign key che identifica il campo \(Code\) della tabella \(Customer\)). Quindi, se volessimo 
 una tabella con ogni cliente e gli ordini ad esso associati, combineremo anche una selezione del tipo :
 \begin{center}
    \(\sigma_{Customer.Code=Order.Customer}(Customer\times Order) =\)
    \begin{tabular}{|l|l|l|l|r|}
        \hline
        Code & Name & CO & Customer  \\
        \hline
        001 & Giancarlo & AX00 & 001 \\\hline
        002 & Sara & AX01 & 002 \\
        \hline
        \end{tabular}
 \end{center}
 È chiaro che, per tabelle con numerosi elementi, il prodotto cartesiano da computare risulta parecchio 
 dispendioso, dato che, se \(R_1\) ha \(n\) elementi, ed \(R_2\) ha \(m\) elementi, il prodotto cartesiano 
 \(R_1\times R_2\) avrà \(n\cdot m\) elementi, e quando si opera su basi di dati con milioni di righe, tale operazione 
 deve essere evitata quando possibile, per questo esiste un'operazione simile al prodotto cartesiano, 
 che seleziona automaticamente le righe da combinare secondo parametri ben precisi.\newpage
 \subsubsection{Join}
 L'operatore join, indicato dal simbolo \(\Join \), seleziona le tuple del prodotto cartesiano 
 che soddisfano una precisa condizione \(C\), ossia che gli attributi con lo stesso nome, debbano 
 avere anche lo stesso valore, è quindi importante che si denominino in maniera precisa gli attributi 
 di relazioni che si vogliono combinare con il join.\begin{equation}
    R_1\Join R_2 = \sigma_C(R_1\times R_2)\end{equation}\begin{equation}
    C = R_1.A_1 = R_2.A_1 \land R_1.A_2 = R_2.A_2 \land ... \land R_1.A_k = R_2.A_k 
\end{equation}
dove \(A_1,A_2...,A_k\) sono gli attributi condivisi con lo stesso nome fra \(R_1\) e \(R_2\).
La tabella risultante, non presenterà due volte gli attributi da \(R_1\) e \(R_2\) con lo stesso 
nome, ma li unirà in un attributo solo. \\Se dovesse 
capitare che le due relazioni non hanno alcun attributo in comune, il risultato sarà il prodotto cartesiano.
\begin{center}
    Siano : 
    \begin{tabular}{|l|l|r|}
        \hline
    \multicolumn{2}{|c|}{\textbf{Customer}}\\
        \hline
        Code & Name  \\
        \hline
        001 & Giancarlo  \\
        \hline
        002 & Sara  \\
        \hline
        003 & Lucia  \\
        \hline
        \end{tabular}
        \begin{tabular}{|l|l|l|r|}
            \hline
        \multicolumn{3}{|c|}{\textbf{Order}}\\
            \hline
            CO & Code & Item  \\
            \hline
            AX00 & 001 & Glue \\
            \hline
            AX01 & 002 & Bricks \\
            \hline
            AX02 & 001 & Shoes \\
            \hline
            \end{tabular}
 Si ha che :\end{center}\begin{center}
    \(Customer\Join Order =\)
    \begin{tabular}{|l|l|l|l|r|}
        \hline
        Customer.Code & Name & CO & Item  \\
        \hline
        001 & Giancarlo & AX00 & Glue \\\hline
        001 & Giancarlo & AX02 & Shoes \\\hline
        002 & Sara & AX01 & Bricks \\
        \hline
        \end{tabular}
 \end{center}
\subsubsection{\(\Theta\)-Join}
Il \(\Theta\)-Join (Che si pronuncia "theta join"), selezione le tuple risultanti dal prodotto cartesiano,
che soddisfino la condizione \(A\Theta B\) dove \(\Theta\in\{<,>,=,\le,ge\}\) ed \(A,B\) sono 
attributi rispettivamente della prima, e seconda relazione (il dominio di \(A\) deve essere lo stesso
del dominio di \(B\)).\begin{equation}
    R_1\Join_{A\Theta B}R_2 = \sigma_{A\Theta B}(R_1\times R_2)
\end{equation}
\subsubsection{Condizioni Negative}
È possibile utilizzare il simbolo \(\lnot C\) per intendere che si vogliono selezionare tutte le 
tuple che non soddisfino la condizione \(C\).
\begin{center} Sia : 
    \begin{tabular}{|l|l|l|r|}
        \hline
    \multicolumn{3}{|c|}{\textbf{Customer}}\\
        \hline
        Code & Name & Town \\
        \hline
        001 & Giancarlo & Roma \\
        \hline
        002 & Sara & Cagliari \\
        \hline
        003 & Nicola & Roma \\
        \hline
        004 & Gianluca & Venezia \\
        \hline
        \end{tabular}
        ho che : \end{center}\begin{center} \(\sigma_{\lnot(Town="Roma")}(Customer)=\)
        \begin{tabular}{ |l|l|l|r| } 
            \hline 002 & Sara & Cagliari \\ 
            \hline 004 & Gianluca & Venezia \\
        \hline
           \end{tabular}
\end{center}
\newpage\subsection{Quantificazione Universale}
Per ora tutti gli operatori che abbiamo visto, implicano una quantificazione esistenziale, quando seleziono 
da una relazione le righe che soddisfano una condizione C, seleziona le righe anche se vi è un altro 
elemento che non soddisfa tale condizione.\\\hphantom{.}\\ Se seleziono tutti i clienti che hanno ordini con 
prezzo superiore a 100 euro, selezionerà tutti i clieni che hanno \textbf{almeno} un ordine 
superiore a 100 euro, non solo quelli che hanno \textbf{esclusivamente} ordini superiori a 100 euro.
\\ Vogliamo scrivere query che implichino la quantificazione universale e non esistenziale, ossia, 
che tale condizione valga "\textbf{per ogni}", e ciò è equivalente a "\textbf{non c'e ne sono tali che}":
\begin{itemize}
    \item I clienti che hanno solo ordini superiori a 100 euro = I clienti che non hanno ordini inferiori o uguali a 100 euro
\end{itemize}
\begin{center}
    Siano : 
    \begin{tabular}{|l|l|r|}
        \hline
    \multicolumn{2}{|c|}{\textbf{Customer}}\\
        \hline
        Code & Name  \\
        \hline
        001 & Giancarlo  \\
        \hline
        002 & Sara  \\
        \hline
        003 & Lucia  \\
        \hline
        \end{tabular}
        \begin{tabular}{|l|l|l|r|}
            \hline
        \multicolumn{3}{|c|}{\textbf{Order}}\\
            \hline
            CO & Code & Price  \\
            \hline
            AX00 & 001 & 90 \\
            \hline
            AX01 & 002 & 120 \\
            \hline
            AX02 & 001 & 200 \\
            \hline
            \end{tabular}
\end{center}
Vogliamo elencare nome e codice dei clienti che hanno fatto ordini esclusivamente con prezzo superiore a 100 euro, quindi, 
\textit{Giancarlo} non sarà incluso, dato che ha fatto si un ordine da 200 euro, ma anche uno da 90 euro.
La nostra query \(Q\) sarà :
\begin{equation}
    Q = Customer-\pi_{Code,Name}(\sigma_{Price\le 100}(Customer \Join Order))
\end{equation}
Quindi, prendere tutti coloro che hanno fatto solo ordini superiori a 100 euro, equivale a prendere 
il totale, e sottrarne coloro che hanno fatto almeno un ordine inferiore o uguale a 100 euro.
\begin{center}
    \(Q = \)
    \begin{tabular}{|l|l|r|}
        \hline
        Code & Name  \\
        \hline
        
        002 & Sara  \\
        
        \hline
        \end{tabular}
\end{center}\subsection{Esempi di Esercizi}\subsubsection{Esempio 1}
Ci sono situazioni in cui dobbiamo combinare una relazione con se stessa per ottenere paia di 
tuple della stessa tabella, vediamo un esempio, si osservino le seguenti relazioni :
\begin{center}
    
        \begin{tabular}{|llll|}
        \hline
        \multicolumn{4}{|c|}{Impiegati}                                                                  \\ \hline
        \multicolumn{1}{|l|}{Name}    & \multicolumn{1}{l|}{Cod} & \multicolumn{1}{l|}{Salario} & CodSup \\ \hline
        \multicolumn{1}{|l|}{Rossi}   & \multicolumn{1}{l|}{C1}  & \multicolumn{1}{l|}{100}     & C3     \\ \hline
        \multicolumn{1}{|l|}{Pirlo}   & \multicolumn{1}{l|}{C2}  & \multicolumn{1}{l|}{200}     & C3     \\ \hline
        \multicolumn{1}{|l|}{Bianchi} & \multicolumn{1}{l|}{C3}  & \multicolumn{1}{l|}{500}     & NULL   \\ \hline
        \multicolumn{1}{|l|}{Verdi}   & \multicolumn{1}{l|}{C4}  & \multicolumn{1}{l|}{200}     & C2     \\ \hline
        \multicolumn{1}{|l|}{Neri}    & \multicolumn{1}{l|}{C5}  & \multicolumn{1}{l|}{150}     & C1     \\ \hline
        \multicolumn{1}{|l|}{Tosi}    & \multicolumn{1}{l|}{C6}  & \multicolumn{1}{l|}{100}     & C1     \\ \hline
        \end{tabular} 
\end{center}
Dove \(CodSup\) identifica il \(Cod\) della riga che ogni impiegato ha come supervisore. Ad esempio, Rossi ha 
come supervisore Bianchi, e Neri ha come supervisore Rossi. Vogliamo trovare tutti gli impiegati che hanno 
lo stipendio superiore o uguale al loro supervisore. Per procedere possiamo combinare tramite il prodotto cartesiano 
la tabella con se stessa, prima però rinominando ogni attributo aggiungendoci una \(C\) davanti, per identificare 
una tabella che è copia dell'altra:
\begin{equation}
    ImpiegatiC = \rho_{Name,Cod,Salario,CodSup\rightarrow CName,CCod,CSalario,CCodSup }(Impiegati)
\end{equation}
Adesso procediamo col combinare tali tabelle, però selezionando esclusivamente le tuple che hanno 
il \(CodSup\) ed il \(CCod\) identici, in modo che avremo una lista dei dipendenti con a destra 
il loro superiore.
\begin{equation}
    \sigma_{CodSup=CCod}(Impiegati \times ImpiegatiC)
\end{equation}
\begin{center}
    \begin{tabular}{|l|l|l|l|l|l|l|l|}
        \hline
        Name  & Cod & Salario & CodSup & CName   & CCod & CSalario & CCodSup \\ \hline
        Rossi & C1  & 100     & C3     & Bianchi & C3   & 500      & NULL    \\ \hline
        Pirlo & C2  & 200     & C3     & Bianchi & C3   & 500      & NULL    \\ \hline
        Verdi & C4  & 200     & C2     & Pirlo   & C2   & 200      & C3      \\ \hline
        Neri  & C5  & 150     & C1     & Rossi   & C1   & 100      & C3      \\ \hline
        Tosi  & C6  & 100     & C1     & Rossi   & C1   & 100      & C3      \\ \hline
        \end{tabular}
\end{center}
Fatto ciò adesso, ci basta selezionare quelli che hanno il salario superiore o uguale a quello del proprio 
supervisore, ossia \(Salario\ge CSalario\) :
\begin{equation}
    \sigma_{(CodSup=CCod)\land(Salario\ge CSalario)}(Impiegati \times ImpiegatiC)
\end{equation}\begin{center}
    \begin{tabular}{|l|l|l|l|l|l|l|l|}
        \hline
        Name  & Cod & Salario & CodSup & CName & CCod & CSalario & CCodSup \\ \hline
        Verdi & C4  & 200     & C2     & Pirlo & C2   & 200      & C3      \\ \hline
        Neri  & C5  & 150     & C1     & Rossi & C1   & 100      & C3      \\ \hline
        Tosi  & C6  & 100     & C1     & Rossi & C1   & 100      & C3      \\ \hline
        \end{tabular}
\end{center}\subsubsection{Esempio 2}
Vediamo adesso un esempio di un altro tipo, si consideri sempre la stessa relazione di prima :
\begin{center}
    
    \begin{tabular}{|llll|}
    \hline
    \multicolumn{4}{|c|}{Imp}                                                                  \\ \hline
    \multicolumn{1}{|l|}{Name}    & \multicolumn{1}{l|}{Cod} & \multicolumn{1}{l|}{Salario} & CodSup \\ \hline
    \multicolumn{1}{|l|}{Rossi}   & \multicolumn{1}{l|}{C1}  & \multicolumn{1}{l|}{100}     & C3     \\ \hline
    \multicolumn{1}{|l|}{Pirlo}   & \multicolumn{1}{l|}{C2}  & \multicolumn{1}{l|}{200}     & C3     \\ \hline
    \multicolumn{1}{|l|}{Bianchi} & \multicolumn{1}{l|}{C3}  & \multicolumn{1}{l|}{500}     & NULL   \\ \hline
    \multicolumn{1}{|l|}{Verdi}   & \multicolumn{1}{l|}{C4}  & \multicolumn{1}{l|}{200}     & C2     \\ \hline
    \multicolumn{1}{|l|}{Neri}    & \multicolumn{1}{l|}{C5}  & \multicolumn{1}{l|}{150}     & C1     \\ \hline
    \multicolumn{1}{|l|}{Tosi}    & \multicolumn{1}{l|}{C6}  & \multicolumn{1}{l|}{100}     & C1     \\ \hline
    \end{tabular} 
\end{center}
Adesso vogliamo trovare l'impiegato che il salario più alto, ma come possiamo fare? Procederemo con il comparare 
ogni impiegato con tutti gli altri, seleziondo esclusivamente quelli che hanno lo stipendio inferiore all'impiegato 
con la quale sono stati comparati (selezioniamo esclusivamente il codice impiegato):\begin{equation}
    Imp2=Imp
\end{equation}
\begin{equation}
    \pi_{Imp.cod}(Imp \Join_{Imp.Salario<Imp2.Salario}Imp2)
\end{equation}
\newpage Ottengo la tabella :
\begin{center}
    \begin{tabular}{|l|}
        \hline
        Impiegati.Cod \\ \hline
        C1            \\ \hline
        C2            \\ \hline
        C4            \\ \hline
        C5            \\ \hline
        C6            \\ \hline
        \end{tabular}
\end{center}
Che rappresenta tutti gli impiegati che hanno qualcuno con lo stipendio superiore al loro. Quindi,
chi non è presente in questa tabella, sarà l'impiegato con lo stipendio più alto. Prendiamo allora tutti 
gli impiegati e sottraiamo ad essi la nostra tabella.
\begin{center}
    \(\pi_{Cod}(Imp)-\pi_{Imp.cod}(Imp \Join_{Imp.Salario<Imp2.Salario}Imp2)=\)\\\hphantom{.}\\=
    \begin{tabular}{|l|}
        \hline
        C1 \\ \hline
        C2 \\ \hline
        C3 \\ \hline
        C4 \\ \hline
        C5 \\ \hline
        C6 \\ \hline
        \end{tabular} \(-\)
        \begin{tabular}{|l|}
            \hline
            C1 \\ \hline
            C2 \\ \hline
            C4 \\ \hline
            C5 \\ \hline
            C6 \\ \hline
            \end{tabular}
            \(=\)
    \begin{tabular}{|l|}
        \hline
        C3 \\ \hline
        \end{tabular}
\end{center}
Quindi l'impiegato di codice \(C3\), ossia \begin{tabular}{|l|l|l|l|}
    \hline
    Bianchi & C3 & 500 & NULL \\ \hline
    \end{tabular} è colui con lo stipendio più alto. 
\section{Design di un Database}
L'obbiettivo è capire come creare uno schema in maniera corretta, immaginiamo di dover creare 
una base di dati per memorizzare le informazioni relative agli studenti di un corso di laurea 
triennale, ed i relativi esami sostenuti.\begin{center}
    \begin{tabular}{|l|l|l|l|l|l|l|l|l|l|l|}
        \hline
        Matricola  & SurN & Name & BirthD & City & Prov & ExCode & ExName & Doc & Date & Grade \\ \hline
        01  & Rossi & Mario & ... & Roma & Roma & 10 & Physics & Pippo & ... & 28 \\ \hline
        02  & Bianchi & Paolo & ... & Tolfa & Roma & 10 & Physics & Pippo & ... & 26 \\ \hline
        01  & Rossi & Mario & ... & Roma & Roma & 20 & Chemistry & Pluto & ... & 27 \\ \hline
        \end{tabular}
\end{center}
È estremamente sbagliato salvare tutti i dati in una sola relazione, per ogni esame, devo salvare 
ogni volta tutti i dati di uno studente, creando ridondanza e spreco di memoria, inoltre se uno studente
non ha sostenuto alcun esame, non apparirà nell'archivio, ed un esame che non è stato sostenuto da 
nessuno risulterà inesistente. È corretto suddividere tale relazione in 3 diverse tabelle :
\begin{center}
    \begin{tabular}{|l|l|l|l|l|r|}
        \hline
        \multicolumn{6}{|c|}{Studenti}   \\\hline
        Matricola  & SurN & Name & BirthD & City & Prov  \\ \hline
        01  & Rossi & Mario & ... & Roma & Roma \\ \hline
        02  & Bianchi & Paolo & ... & Tolfa & Roma  \\ \hline
        01  & Rossi & Mario & ... & Roma & Roma  \\ \hline
        \end{tabular}
        \begin{tabular}{|l|l|r|}
            \hline
            \multicolumn{3}{|c|}{Corso}   \\\hline
            ExCode  & ExName & Doc  \\ \hline
            10  & Physics & Pippo \\ \hline
            20  & Chemistry & Pluto\\ \hline
            \end{tabular}
\end{center}
\begin{center}
    \begin{tabular}{|l|l|l|r|}
        \hline
        \multicolumn{4}{|c|}{Esami}   \\\hline
        Matricola  & ExCode & Data & Grade  \\ \hline
        01  & 10 & ... & 28  \\ \hline
        01  & 20 & ... & 27  \\ \hline
        02  & 10 & ... & 26  \\ \hline
        \end{tabular}
\end{center}
\subsection{Definizioni Formali}
Uno \textbf{schema relazionale}, che si denota con \(R\) è un insieme di attributi del tipo:\begin{center}
    \(R=\{A_1,A_2,A_3...,A_k\}\)
\end{center}
Solitamente un sotto insieme di attributi di \(R\) viene indicato con le lettre \(X\) oppure \(Y\),
e l'unione si può denotare \(XY\equiv X\cup Y\). 
Data una relazione \(R\), una \textbf{tupla} \(r\) è una funzione che associa ad ogni attributo, 
un valore appartenente al suo dominio : \begin{center}
    \(
      R=\{nome,cognome\}\)   \(r[nome]=marco, r[cognome]=rossi  
    \)
\end{center}
Preso \(X\subset R\), diciamo che due tuple \(r_1,r_2\) \textbf{coincidono} su \(X\) se :\begin{center}
    \(\forall A\in X,r_1[A]=r_2[A]\)
\end{center}
\begin{center}
    \begin{tabular}{|l|l|r|}
        \hline
        \multicolumn{3}{|c|}{Corso}   \\\hline
        Nome  & Cognome & Voto  \\ \hline
        Paolo  & Rossi & 29 \\ \hline
        Mario  & Rossi & 29\\ \hline
        \end{tabular}
        Le due tuple coincidono su \(\{\)Cognome,Voto\(\}\)
\end{center}
Un \textbf{istanza} su uno schema relazionale \(R\) è l'insieme di tutte le tuple su \(R\).
Una \textbf{dipendenza funzionale} su \(R\) è una coppia ordinata \(X,Y\) di sotto-insiemi di \(R\), ossia 
di attributi di \(R\), che si denota con \(X\rightarrow Y\), ed indica che :\begin{center}
    per ogni coppia di tuple \(r_1,r_2\), vale che \(r_1[X]=r_2[X]\implies r_1[Y]=r_2[Y]\)
\end{center}
E si dice \(X\) \textit{determina} \(Y\), dove \(X\) è detto determinante ed \(Y\) dipendente. Le dipendenze 
funzionali esprimono dei \textit{vincoli}. Uno schema \(R\) può avere un \textbf{insieme di dipendenze funzionali}
denotato con \(F\) :\begin{center}
    \(F=\{A\rightarrow B,D\rightarrow B...,G\rightarrow H\}\)
\end{center} se un istanza soddisfa ogni dipendenza di \(F\), ossia tutte le dipendenze funzionali, è 
detta istanza \textbf{legale}. Le dipendenze godono di una relazione di \textit{transitività}, ossia,
se \(A\rightarrow B\) e \(B\rightarrow C\), allora \(A\rightarrow C\), non è però necessario aggiungere 
quest'ultima all'insieme \(F\), dato che è implicita, e vogliamo mantenere \(F\) il più piccolo possibile. 
Ci sono quindi dipendenze funzionali che sono soddisfatte per ogni istanza della relazione, che però 
non è necessario includere in \(F\), un altro esempio :\begin{center}
    se \(taxCode\rightarrow name,surname\) \\
    è ovvio che \(taxCode\rightarrow name\)  e  \(taxCode\rightarrow surname\)
\end{center}
Ma queste due ultime non saranno incluse in \(F\). Tali dipendenze sono dette \textbf{banali}, ad 
esempio, se \(Y\subset X\), se \(A\rightarrow X\) ovviamente \(A\rightarrow Y\). Il numero delle dipendenze 
banali è \textit{esponenziale} :\begin{center}
    \(
    X\rightarrow Y \iff \forall A\in Y, X\rightarrow A    
    \)
\end{center} 
Tutte le dipendenze banali che non è necessario includere in \(F\), si trovano in un insieme 
più grande detto \textbf{chiusura di }\(F\), che si denota con \(F^+\), contenente tutte le 
dipendenze funzionali soddisfatte da un'istanza, è chiaro che \(F\subseteq F^+\).

Passiamo alla definizione di \textbf{chiave}. Un sotto-insieme di attributi \(K\) è detto chiave se :\begin{itemize}
    \item \(K\rightarrow R \in F^+\)
    \item \(\nexists K'\subset K | K'\rightarrow R \in F^+\)
\end{itemize}
Detto in maniera meno formale, non esisteranno due tuple di una relazione con gli stessi valori 
per le chiavi. Nei linguaggi come \textit{SQL}, fra gli attributi della chiave se ne definisce uno 
in particolare detto \textbf{chiave primaria},che non può assumere valore \code{NULL}.
\end{document}