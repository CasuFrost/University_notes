\documentclass[12pt, letterpaper]{article}
\usepackage{graphicx} % Required for inserting images
\usepackage{hyperref}
\usepackage{xcolor}
\usepackage{amssymb}
\usepackage{amsmath}
\usepackage[english]{babel}
\usepackage{nicefrac, xfrac}
\usepackage{tikz}
\newcommand{\Z}{{\mathbb Z}}
\newcommand{\R}{{\mathbb R}}
\newcommand{\N}{{\mathbb N}}
\newcommand{\Sn}{{\mathcal S_n}}
\newcommand{\An}{{\mathcal A_n}}
\newcommand{\dimo}{{\text{\textbf{Dimostrazione }:}}}
\newcommand{\mcm}{{\text{mcm}}}
\newcommand{\spaz}{{\text{\hphantom{aa}}}}
\newcommand{\MCD}{{\text{MCD}}}
\newcommand{\supp}{{\text{Supp}}}
\newcommand{\acc}{\\\hphantom{}\\}
\newcommand{\aut}{{\text{Aut}}}
\newcommand{\cen}{{\text{Centro}}}
\newcommand{\norm}{{\unlhd}}
\newcommand{\ciclS}{{\left \langle }}
\newcommand{\ciclE}{{\right \rangle }}
\usetikzlibrary{positioning}
\usepackage[paper=a4paper,left=20mm,right=20mm,bottom=25mm,top=25mm]{geometry}
\begin{document}
\textbf{Foglio 7 esercizio 2}\\\hphantom{}\\
Voglio prima di tutto dimostrare che \(G\) sia un sottogruppo, prima però serve un osservazione : Se 
\(G\) è un gruppo ogni elemento \(ax+c\in G\) ha un inverso, ed esso è : \((ax+c)^{-1}=\dfrac{x-c}{a}\).\\
Verifico che \(G\) sia un sottogruppo di tutte le bigezioni in \(\R\) :\begin{equation}
    (a'x+c')\circ(a''x+c'')^{-1}= (a'x+c')\circ(\dfrac{x-c''}{a''})=a'(\dfrac{x-c''}{a''})+c'=\dfrac{a'x-c''a'}{a''}+c
\end{equation}
\begin{equation}
    =(\dfrac{a'x}{a''}-\dfrac{a'c''}{a''})+c'=(x\dfrac{a'}{a''}-\dfrac{a'c''}{a''})+c'=x(\dfrac{a'}{a''}-\dfrac{a'c''}{a''})+c'\in G 
\end{equation}
quindi \(G\) è un sottogruppo, e dimostro che non è commutativo : 
\begin{equation}
    \begin{cases}
        (a'x+c')\circ(a''x+c'')=a'(a''x+c'')+c'\\
        (a''x+c'')\circ(a'x+c')=a''(a'x+c')+c''
    \end{cases}\implies (a'x+c')\circ(a''x+c'')\ne( a''x+c'')\circ(a'x+c')
\end{equation}
Considero adesso un sottogruppo particolare di \(G\), ossia \(T=\{f_{1,c},c\in\R\}=\{x+c, c\in \R\}\), dimostro che 
è un sottogruppo : \begin{equation}
    (x+c')\circ(x+c'')^{-1}= (x+c')\circ(x-c'')=(x-c'')+c'=x+(c'-c'')\in T
\end{equation}
Inoltre definisco le classi laterali sinistre di \(T\), ossia : 
\(gT=\{g\circ t, g\in G, t\in T\}\) che sono tutte le funzioni del tipo : \begin{equation}
    g=ax+c\implies gT=\{(ax+c)\circ (x+c'),(ax+c)\circ (x+c''),(ax+c)\circ (x+c''')...\}
\end{equation}
Le classi laterali destre di \(T\), ossia : 
\(Tg=\{t\circ g, g\in G, t\in T\}\) che sono tutte le funzioni del tipo : \begin{equation}
    g=ax+c\implies gT=\{(x+c')\circ(ax+c) ,(x+c'')\circ(ax+c),(x+c''')\circ(ax+c)...\}
\end{equation}
Adesso noto che :\begin{equation}
    \begin{cases}
        (x+c')\circ (ax+c)=ax+c+c'
        \\(ax+c)\circ (x+c')=ax+c+ac'
    \end{cases}
\end{equation}
(\color{red}dimostrazione omessa\color{black})Noto che le classi laterali destre e sinistre sono le stesse, quindi 
\(T\) è un sottogruppo normale, e posso definire il gruppo di tutte le classi laterali, ossia il gruppo 
quoziente \(\nicefrac{G}{T}\), con l'operazione \(gT*hT=(g\circ h)T\).
\\Adesso, definisco un'applicazione suriettiva \(\varphi : G\rightarrow \R\backslash\{0\}\), tale che : \begin{equation}
    \varphi(ax+c)=a
\end{equation}
E definisco il suo nucleo \begin{equation}
    Ker\varphi = \{ax+c|\varphi(ax+c)=1\iff a=1\}=\{x+c, c\in \R\}
\end{equation}
\textbf{Osservazione fondamentale} : Noto che \(Ker\varphi = T\)! Definisco \(\pi\) la proiezione canonica :\begin{equation}
    \pi(ax+c)=(ax+c)T
\end{equation} 
Noto che per il teorema fondamentale di omomorfismo di gruppi, esiste un \textbf{unico isomorfismo} \(F:\nicefrac{G}{Ker\varphi}\rightarrow \R\backslash\{0\}\) 
tale che \(\varphi=\pi\circ F\), essendo  \(Ker\varphi = T\), so che \(\nicefrac{G}{T}\) è isomorfo a \(\R\backslash\{0\}\).

\end{document}