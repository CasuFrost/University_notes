\documentclass[1pt, letterpaper]{article}
\usepackage{graphicx} % Required for inserting images
\usepackage{hyperref}
\usepackage[table,xcdraw]{xcolor}
\usepackage{amssymb}
\usepackage{amsmath}
\usepackage[english]{babel}
\usepackage{nicefrac, xfrac}
\usepackage{tikz}
\definecolor{light-gray}{gray}{0.95}
\newcommand{\code}[1]{\colorbox{light-gray}{\texttt{#1}}}
\newcommand{\K}{{\mathbb K}}
\newcommand{\Z}{{\mathbb Z}}
\newcommand{\R}{{\mathbb R}}
\newcommand{\N}{{\mathbb N}}
\newcommand{\Sn}{{\mathcal S_n}}
\newcommand{\An}{{\mathcal A_n}}
\newcommand{\E}{{\mathcal E}}
\newcommand{\B}{{\mathcal B}}
\newcommand{\dimo}{{\text{\textbf{Dimostrazione }:}}}
\newcommand{\mcm}{{\text{mcm}}}
\newcommand{\rg}{{\text{rg}}}
\newcommand{\ve}{{\bar v}}
\newcommand{\spaz}{{\text{\hphantom{aa}}}}
\newcommand{\MCD}{{\text{MCD}}}
\newcommand{\supp}{{\text{Supp}}}
\newcommand{\acc}{\\\hphantom{}\\}
\newcommand{\aut}{{\text{Aut}}}
\newcommand{\Span}{{\text{Span}}}
\newcommand{\End}{{\text{End}}}
\newcommand{\cen}{{\text{Centro}}}
\newcommand{\norm}{{\unlhd}}
\newcommand{\ciclS}{{\left \langle }}
\newcommand{\ciclE}{{\right \rangle }}
\newcommand{\fdot}{{\(\bullet\) }}
\newcommand{\boxedMath}[1]{\begin{tabular}{|c|}\hline \texttt{#1} \\ \hline\end{tabular} :}
\usetikzlibrary{positioning}

\usepackage[paper=a4paper,left=0.3mm,right=0.3mm,bottom=0.3mm,top=0.3mm]{geometry}
\title{Algebra}
\author{Marco Casu}
\date{\vspace{-5ex}}
\begin{document}
\newcommand{\omo}[1]{\(\varphi\)}

\fdot Una relazione di equivalenza è simmetrica, riflessiva e transitiva. é di ordine parziale se è  riflessiva e transitiva ed antisimmetrica (\(a\rho b\land b\rho a \implies a=b\)).
\fdot \textbf{Risoluzione equazione diofantea} : si ha \(ax + by = c\) (1)  Bisogna prima verificare che l’equazione sia risolvibile, si calcoli quindi
\(MCD(a, b) = d\), se esso divide c, l’equazione ammette soluzione. (2)Usare l’algoritmo euclideo per trovare un’identità di Bèzout per \(d\), esprimendolo
nella forma \(d = ax_0 + by_0\), utilizzeremo proprio tali coefficenti \((x_0, y_0)\). (3) Considero \((\tilde x,\tilde y)=(\frac{c}{d}\cdot x_0,\frac{c}{d}\cdot y_0)\) (4) Le soluzioni saranno 
\((\tilde x+k\cdot \frac{b}{d},\tilde y -k\cdot \nicefrac{a}{d} )\).\fdot Siano \(a=p_1^{h_1}p_2^{h_2}\dots p_s^{h_s}\) e \(b==p_1^{k_1}p_2^{k_2}\dots p_s^{k_s}\), allora 
\(MCD(a,b)=p_1^{m_1}p_2^{m_2}\dots p_s^{m_s}\) e \(mcm(a,b)=p_1^{M_1}p_2^{M_2}\dots p_s^{M_s}\) con \(m=\min(h_i,k_i)\) e \(M=\max(h_i,k_i)\). \fdot \textbf{Proprietà anello} : (1) \(a\cdot(-b)=-(ab)=(-a)\cdot b\)
(2) \((-a)\cdot (-b) = ab\) (3) \(a\cdot(b-c)=(a\cdot b)-(a\cdot c)\).\fdot \textbf{Teorema}: Sia A un anello unitario con finiti elementi e privo di divisori dello zero.
Allora A è un anello di divisione\fdot \textbf{Costruzione di \(\Z_n\)} : Considero la relazione \(a\sim b\iff a-b\) è divisbile per \(n\). L'insieme \(\Z_n:=\nicefrac{\Z}{\sim}\) 
è l'insieme delle classi di equivalenza. \fdot Una \textbf{congruenza lineare} del tipo \(ax=b\mod n\) è equivalente al risolvere l'eq. diofantea \(ax+ny=b\). Un'eq. congruenziale ammette soluzione se e solo 
se \(MCD(a,n)\) divide \(b\). La \textbf{funzione di Eulero} associa ad \(a\) il numero degli elementi coprimi con \(a\) minori di \(a\). Se \(p\) è primo, allora \(\varphi(p^h)=p^h-p^{h-1}\). Teo di Eulero : Se 
\(MCD(a,n)=1\) allora \(a^{\varphi(n)}=1\mod n\). Picc. Teo di Fermat : Se \(p\) è primo \(\forall a\spaz a^p=a\mod p\). \fdot \textbf{Costruzione} di \(\Z\) : si considera \(\N\times \N\) e la relazione 
\((n,m)\sim(n',m')\iff n+m'=m+n'\) Si ha che \(\Z=\nicefrac{\N\times \N}{\sim}\). il prodotto : \([(n,m)]\cdot[(n',m')]=[(nn'+mm',nm'+n'm)]\). Ogni \(a,b\ne0\in \Z \) esistono unici \(q,r\) tali che 
\(a=bq+r\) con \(0\le r<|b|\). \fdot \textbf{Teo. cinese} un sistema cinese  ha gli argomenti dei moduli co-primi fra loro e l'incognita ha come coefficiente 1 (\(x=c_k \mod r_k)\). Siano \(r_1,r_2\dots r_s\) gli argomenti dei moduli, 
sia \(R=r_1\cdot...r_s\) ed \(R_k=\frac{R}{r_k}\). Sia \(t_k\) la sol di \(R_kt_k+r_kg_k=1\), e \(\bar x_k=c_kt_k\). L'unica soluzione del sistema è \(\displaystyle\sum_{i=1}^s\bar x_iR_i\). \fdot Un equazione 
in un sistema cinese \(x=c \mod rs\), se \(MCD(r,s)=1\) diventa due equazioni \(\begin{cases}
    x=c\mod r\\x=c\mod s
   \end{cases}\).
   \fdot \textbf{Costruzione} di \(\mathbb{Q}\) : Si considera \(\Z\times \Z\backslash \{0\}\) con la relazione \((a,b)\sim(c,d)\iff ad=bc\). \(\mathbb{Q}=\nicefrac{\Z\times \Z\backslash \{0\}}{\sim}\). Il prodotto è banale 
   si moltiplicano le coordinate, somma : \([(a,b)]+[(c,d)]=[(ad+bc,bd)]\). \fdot \color{red}\textbf{Criterio sottogruppo normale} \color{black} Sia \(h\in H\), \(H\norm G\iff a*h*(a^{-1})\in H \forall a\in G\) \fdot sugli \textbf{ordini},
   si ha che \(o(g^s)=\dfrac{mcm(o(g),s)}{s}\). La partizione fornita dalle classi laterali stabilisce una relazione \(a\rho_S b\iff \exists g\in G|a\in gH\land b\in gH\). 
   \fdot sia \(\varphi\) un omomorfismo : \(o(\varphi(g)) \) divide \(o(g)\), se è iniettivo  \(o(\varphi(g))=o(g)\). 
   \fdot \textbf{Gruppo Simmetrico} due perm. sono coniugate se hanno la stessa struttura ciclica. Decomposizione in trasp : \((a_1\spaz a_2\spaz a_3...a_n)=(a_1\spaz a_n)\dots(a_1\spaz a_3)(a_1\spaz a_2)\).
   Se \(\sigma_i(a)=b\land \tau(a)=s\) allora \(\tau\sigma_i\tau^{-1}(s)=\tau\sigma_i(a)=\tau(b)\). L'\textbf{ordine di una permutazione}
   è uguale al minimo comune multiplo delle lunghezze dei cicli che la compongono.
   
   \fdot \textbf{Teo. fond. omomorfismo} : \(f:G\rightarrow G'\) un omomorfismo. sappiamo che \(Kerf\norm G\), consideriamo il gruppo quoziente \(\nicefrac{G}{Kerf}\). Sia \(\pi :G\rightarrow \nicefrac{G}{Kerf}|
   \pi(g)=gKerf\). Esiste unico isomorfismo \(F:\nicefrac{G}{Kerf}\rightarrow Im(G)\) tale che \(f=F\circ \pi\). 
   
   
   
   
   \newpage
 \fdot Sviluppo di laplace : \(\det(A)=\displaystyle\sum_{k=1}^n(-1)^{i+k}a_{i_k}\det(A_{(i,k)})\). \(A_{(i,k)}\) è la matrice \(A\) senza la riga \(i\) e la colonna \(j\). \fdot Sia \(S\in M_{n,m}\) una matrice a 
   scala di \(n\) righe ed \(m\) colonne, di rango \(r\), il sistema \(S\bar x=\bar b\) ha soluzione se e solo se le ultime \(m-r\) coordinate di \(\bar c\) sono 0, ed lo spazio delle soluzioni di \(S\bar x = \bar 0\) ha 
   dimensione \(n-r\). \fdot La matrice associata a T (applic. lineare) con scelta di basi \(\B=\{b_1\dots,b_n\}\) ed \(\E=\{e_1\dots,e_m\}\) e la matrice che ha come j-ma colonna le coordinate di \(T(b_j)\) 
   nella base \(\E\), ed è una matrice di di \(m\) righe e \(n\) colonne. \fdot Due matrici \(A,B\) sono \textbf{simili} se  \(\exists C|A=C^{-1}BC\). Se \(M_{\B,\B}\) è una matrice associata ad un'applicazione, 
   nelle basi \(\B\) in partenza e \(\B\) in arrivo, allora è \textit{simile} alla matrice \(M_{\E,\E}\), mi basta trovare \(M_{\E,\B}\) e si ha che \(M_{\B,\B}=M_{\E,\B}^{-1}\cdot M_{\E,\E}\cdot M_{\E,\B}\).
   \fdot \textbf{proprietà del determinante} : 
   (1) - se \(A_i=A_j\implies\det(A_1..,A_i..,A_j..,A_n)=0\) (2) - \(\det(A_1..,\lambda A_i..,A_n)=\lambda\det(A_1.., A_i..,A_n)\) (3) - \(\det(A_1.., A_i+A_j..,A_n)=
   \det(A_1.., A_i..,A_n)+\det(A_1.., A_j..,A_n)\).
   ne seguono : (i)- 
   \(\det(A_1..\bar 0,.. A_n)=0\) (ii)- \(\det(A_1.. A_i,.. A_j,.. A_n)=\det(A_1.. A_i+\lambda A_j,.. A_j,.. A_n)\) (iii)- \(\det(A_1.. A_i,.. A_j,.. A_n)=(-1)\det(A_1.. A_j,.. A_i,.. A_n)\).
   \fdot \textbf{Teo. di Rouché Capelli} Sia \(A\) una mat. \(n\times n\) e \(A\bar x=\bar b\) un sistema. Il sistema ha soluzione solo se \(\rg(A)=\rg(A|\bar b)\). Ammette unica soluzione solo se \(\rg(A)=n\). 
   \newpage $$\begin{bmatrix}
    1&4&-1& 0\\ 0&-3&2&1\\ 0&0&0&0 \\ 0&0&0&0
    \end{bmatrix}$$
 \end{document}
